%
%
\documentclass[%
  color,
  epsf,
  url,
  amssymb,
  %slidesonly,%  Try notes or notesonly instead.
  %notes,%      Use instead of slidesonly to typeset the notes.
  %notesonly,%  Use instead of slidesonly to typeset notes and slides.
  %semcolor,%   Try me if using PSTricks.
  semhelv,%    Try me if using a PostScript printer.
  %article,%    Try me.
  portrait,%   Try me.
  %sem-a4,%     Try me if using A4 paper.
  semlayer%     This must be included, but you need the semcolor option to
  ]{seminar}

\slidesmag{4}
\articlemag{1}

%\twoup                     % Try me for twoup printing.

\usepackage{epsf}
\usepackage{url}

%\portraitonly              % To print only portrait slides
%\landscapeonly             % To print only landscape slides

%\notslides{\ref{questions}-7,1}   %Try me: The slides are omitted.
%\onlyslides{\ref{questions}-7,1}  %Try me: Only these slides are included.
%\onlynotestoo                     %Try me: For selecting notes as well.

%\topmargin=1in
%\textwidth=9in

%\setlength{\slidetopmargin}{2in}

%\slidetopmargin=2in


\colorlayers{red,blue}      % Try deleting this if using the semcolor option,
                            % to get \blue and \red to use PostScript color.

\rotateheaderstrue          % Try this out if using rotation macros.
\newcommand{\Past}         {\stackrel{\leftarrow} {S}}
\newcommand{\Future}       {\stackrel{\rightarrow}{S}}
\newcommand{\all}{\stackrel{\leftrightarrow}{S}}
\newcommand{\pr}{\rm Pr}
\newcommand{\past}{ \,\stackrel{\leftarrow}{S} \,}
\newcommand{\future}{\, \stackrel{\rightarrow}{S}\,}


\title{ {\bf Some Foundations in Complex Systems}\\ Tools and
Concepts}  
\author{ {\bf David P. Feldman} }
\date{July 5--11, 2004}

\newcommand{\sref}[1]{SLIDE \ref{#1}}
\newcommand{\heading}[1]{\begin{center}\large\bf #1\end{center}}

\renewcommand{\slidetopmargin}{1.5in}

\newpagestyle{MH}
  {SFI Summer School, Qingdao China, July 2004\hfil\thepage}{Dave
Feldman \hspace{3cm} {\url{http://hornacek.coa.edu/dave}}\hfil}
\pagestyle{MH}

\begin{document}


\maketitle    % This won't show up when \onlynotestoo is in effect.
\begin{center}
{\large
College of the Atlantic \\ and \\
Santa Fe Institute} \\
\end{center}
\begin{center}
{\large \tt \bf
dave@hornacek.coa.edu \\
\smallskip
http://hornacek.coa.edu/dave/ }\\
\end{center}

%\hspace{3mm}\\
%\begin{center}
%{\large {\bf Collaborators:}}\\
%Jim Crutchfield (Santa Fe Institute)\\
%%Susan McKay (University of Maine)
%\end{center}

\begin{slide}
  \ifslidesonly              % Title slide only for slidesonly selection.
    \maketitle
    \addtocounter{slide}{-1}
    \slidepagestyle{empty}
  \fi
\end{slide}


% **********************************************************************
\begin{slide*}
\centerslidesfalse
\begin{center}
{\bf CSSS Thoughts and Advice}
\end{center}

\begin{enumerate}

\item I attended the CSSS in Santa Fe in 1996.

\item The most interesting and valuable parts of the summer school
will probably be the discussions you have with other students and the
work you do in working groups.  The lectures are just starting point.

\item The CSSS is an amazing opportunity to interact with people you
might now normally interact with.  Take advantage of this.  

\item Think of your role as both a learner and a teacher. 

\item Don't spend too much energy worrying about the definition of
complexity or complex systems.  (You wouldn't go to a history
conference and spend a month debating what history is.)

\item Working across disciplines can be challenging.  Different
disciplines have different vocabularies and underlying assumptions.
Be patient---it's worth it.

\end{enumerate}


\end{slide*}
% **********************************************************************



% **********************************************************************
\begin{slide*}
\centerslidesfalse
\begin{center}
{\bf Comments about my series of lectures}
\end{center}


\begin{enumerate}

\item I am trained as a theoretical physicist.  So my approach will be
somewhat physics-centric. 

\item There is a lot of material that I'm trying to cover, so I'll
move quickly and at times will skip some topics that I really
shouldn't skip. 

\item I will be covering some topics that some of you know very well,
and others topics that you don't know that well.  This means there
will be times when the lectures seem too slow or too fast.  I don't
know how to avoid this.  This will probably be the case for all the
lectures this month.

\item Please, ask questions during the lectures!

\item Please, ask questions after the lectures!

\item I'd also welcome comments, critique, and conversation. 

%\item I'd also welcome comments during my lectures; you may be more of
%an expert than me on some of the topics.

\item My aim is to give a somewhat opinionated survey of tools,
methods, and ideas from complex systems.  

\item We can't cover anything, so I've chosen what I think is most
important and what you're less likely to get elsewhere. 

\end{enumerate}


\end{slide*}
% **********************************************************************




% **********************************************************************
\begin{slide*}
\centerslidesfalse
\begin{center}
{\bf Complex Systems?}
\end{center}
%{\bf Some initial thoughts about complex systems}



I'm not interested in a strict definition of complex systems.
However, it seems to me that most things we'd think of as a complex
system share many of the following features:

\begin{enumerate}


\item Unpredictability.  A perfectly predictive theory is rarely
possible. 

\item Emergence: Systems generate patterns that are not part of the
equations of motion:  {\em emergent phenomena}. 

\item Interactions: The interactions between a system's components
play an important role. 

\item Order/Disorder:  Most complex systems are simultaneously ordered
and disordered. 

\item Heterogeneity:  Not all the elements that make up the system are
identical.  

\end{enumerate}

\end{slide*}

% **********************************************************************
% **********************************************************************
\begin{slide*}
\centerslidesfalse
\begin{center}
{\bf Phenomena and Topics}
\end{center}

\begin{itemize}
\item Another way to approach a definition of complex systems is to
list the things that people think are complex systems:
\begin{itemize}

\item Immune system, ecosystems, economies, auction markets,
evolutionary systems, the brain ...

\item Critical phenomena/phase transitions, chaotic dynamics, fractals
and power laws, natural computation ... 

\end{itemize}

\item This amounts to saying:  complex systems are what complex
systems people study.

\item This does have a nice internal consistency.

\item In my opinion, what gets included as part of a discipline is
often a frozen accident. 

\end{itemize}




\end{slide*}

% **********************************************************************
% **********************************************************************
\begin{slide*}
\centerslidesfalse
\begin{center}
{\bf Tools}
\end{center}

Most tools and techniques for complex systems will thus need to:
\begin{enumerate}

\item Measure unpredictability, distinguish between different sorts of
unpredictability, work with probabilities

\item Have tools for measuring and discovering pattern, complexity,
structure, emergence, etc. 

\item Be inferential; be inductive as well as deductive.  Must infer
from the system itself how it should be represented.   

\item Be interdisciplinary; combine methods, techniques, and areas of
study from different fields

\end{enumerate}

\end{slide*}

% **********************************************************************
% **********************************************************************
\begin{slide*}
\centerslidesfalse
\begin{center}
{\bf Models}
\end{center}

We will need model systems upon which to try out tools and
techniques:  ``fruit flies'' and ``lab rats'' for complex systems.
Many are commonly used
\begin{enumerate}
\setlength{\itemsep}{-2mm}
    \item Logistic equation
    \item Random networks
    \item Ising models
    \item Cellular automata
    \item etc.\\
\end{enumerate}

\begin{itemize}

\item Also, one usually has to build a model, or choose a representation, of
a phenomena before one can study it.  

\item It is important to choose model classes carefully.

\item We will also need to think about how to infer models from data.

\item Models, by definition, ignore many aspects of the phenomena
being modeled.

\item There are always choices (sometimes hidden) made when building a
model. 

\end{itemize}

\end{slide*}

% **********************************************************************

% **********************************************************************
\begin{slide*}
\centerslidesfalse
\begin{center}
{\bf Complexity: Initial Thoughts}
\end{center}

\begin{itemize}

\item  The complexity of a phenomena is generally understood to be a
measure of how difficult it to describe it.

\item But, this clearly depends on the language or representation
used for the description.  

\item It also depends on what features of the thing you're trying to
describe.  

\item There are thus many different ways of measuring complexity.  I
will aim to discuss a bunch of these in my lectures. 

\item Some important, recurring questions concerning complexity
measures: 
\begin{enumerate}
   \item What does the measure tell us?
   \item Why might we want to know it?
   \item What representational assumptions are behind it?
\end{enumerate}

\end{itemize}

\end{slide*}

% **********************************************************************


% **********************************************************************
\begin{slide*}
\centerslidesfalse
\begin{center}
{\bf Outline}
\end{center}

\begin{enumerate}

    \item Introductory remarks.  
    \item Introduction to Dynamical Systems and Chaos, Part
    I. Terminology, definition of chaos.      
    \item Introduction to Dynamical Systems and Chaos, Part II 
    \item Information Theory, Part I 
    \item Information Theory, Part II 
    \item Computation Theory, Part I: Automata and Computational Hierarchy 
    \item Computation Theory, Part II: Universal Turing Machines and
    Computational Complexity 
    \item Measures of Complexity, Part I: Computational Mechanics 
    \item Measures of Complexity, Part II: Survey of other Complexity
    Measures  
    \item Conclusions 
\end{enumerate}

\end{slide*}
% **********************************************************************


% **********************************************************************
\begin{slide*}
\centerslidesfalse
\begin{center}
{\bf Goals}
\end{center}

\begin{enumerate}

\item  Present some tools, models, paradigms that are useful in
complex systems. 

\item Discuss the applicability and un-applicability of these various
tools.  

\item Provide references and advice so you can learn more about these
topics if you wish. 

\item Present some thoughts about what makes the study of complex
systems similar too, and different from, other types of science. 

\item Have fun. 

\end{enumerate}

\end{slide*}
% **********************************************************************



\end{document}

