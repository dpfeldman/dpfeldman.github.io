

How does nature self-organize and how can scientists discover such
organization?  Is there an objective notion of pattern, or is the
discovery of patterns a purely subjective process?  And what
mathematical vocabulary is appropriate for describing and quantifying
pattern, structure, and organization?  This dissertation compares and
contrasts the way in which statistical mechanics, information theory,
and computational mechanics address these questions.



After an in-depth review of the statistical mechanical, information
theoretic, and computational mechanical approaches to structure and
pattern, I present exact analytic results for the excess entropy and
$\epsilon$-machines for one-dimensional, finite-range discrete
classical spin systems.  The excess entropy, a form of mutual
information, is an information theoretic measure of apparent spatial
memory.  The $\epsilon$-machine---the central object of computational
mechanics---is defined as the minimal model capable of statistically
reproducing a given configuration, where the model is chosen to belong
to the least powerful model class(es) in a stochastic generalization
of the discrete computation hierarchy. 


These results for one-dimensional spin systems
demonstrate that the measures of pattern from information 
theory and computational mechanics differ from known thermodynamic and
statistical mechanical functions. Moreover, they capture important
structural features that are otherwise missed. In particular, the
excess entropy serves to detect ordered, low entropy 
density patterns. It is superior in many respects to other functions
used to probe the structure of a distribution, such as
structure factors and the specific heat.   More generally,
$\epsilon$-machines are seen to be the most direct approach to
revealing the group and semigroup symmetries possessed by 
the spatial patterns and to estimating the minimum amount of memory
required to reproduce the configuration ensemble, a quantity known as
the {\em statistical complexity}.  It is shown that the information
theoretic and computational mechanical analyses of spatial patterns
capture the intrinsic computational capabilities embedded in spin
systems---how they store, transmit, and manipulate configurational
information to produce spatial structure. 


Finally, several approaches to generalizing the excess entropy and
$\epsilon$-machines to apply to multi-dimensional configurations are
put forth.  These measures are then calculated for a few simple,
two-dimensional patterns.  

