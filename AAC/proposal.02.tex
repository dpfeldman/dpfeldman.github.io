\documentclass[12pt]{article}
\oddsidemargin=-0.25in
\evensidemargin=-0.25in
\textwidth=7.00in
%\topmargin=-.75in
%\textheight=9in
\usepackage{draftcopy}
\usepackage{url}
 
\renewcommand{\arraystretch}{1.0}
 
\begin{document}
\pagestyle{empty}
 
\begin{center}
{\Large {\bf DRAFT Five-Year Academic Plan}} \\
\medskip
Drafted by dpf\\
\medskip
\today
\end{center}

\section{Introduction}

\begin{itemize}

\item {\bf This is a living, flexible document.  It will be
reviewed formally by the faculty meeting and the Academic Affairs
Committee every winter term.} 

\item This document is based on discussions at faculty meetings, the
academic affairs committee, the Fall 2004 faculty retreat, and resource
area discussions. 

\end{itemize}


\section{Numbers, Growth, and Faculty Positions}

\begin{itemize}

\item We expect to grow to around 300 FTE students in around five
years. 

\item  To meet course and advising needs we need to add a bare
minimum of 4.5 FTE faculty.  I believe that a more comfortable
increase in faculty FTE is 7 to 8.   

\item Increasing faculty by 7.5 will decrease the percentage of
courses offered visiting faculty to around 18\%.  This percentage is
currently at 25\%. 

\item For details on FTE and visiting class percentages, including
comparisons with other schools, see appendix \ref{numbers}.  I would
be happy to discuss the details of my calculations with anyone who is
interested. 

\end{itemize}


\section{Top Priorities}

The following are top priorities for academic growth, assuming growth
to 300 students.  Within this category, items are {\em not}
prioritized. 

\begin{itemize}
\setlength{\itemsep}{-1mm}
\item Foreign Languages
\item Wiggins chair in government and polity (starts at 0.5 time)
\item Geology
\item Food Systems (0.6 time teaching)
\item Add visiting instructorships (ensemble directors and music
specialists) in music and other areas of arts and design
\item Regularize visiting course offerings in areas in which we know
there will be ongoing need.  (See section \ref{lectureships} below.)
\item Asian or African Studies
\item Additional arts and design position, details to be determined 

\end{itemize}

\subsection{Discussion}

\begin{enumerate}

\item In the event of retirement or resignation by current faculty,
the default assumption is that the position will continue without
substantive changes.  A possible exception to this is the position
currently held by Elmer Beal.  Elmer's contributions to the curriculum
could be viewed as being ``replaced'' by the Food Systems and the Asian
Studies hires.  Also, Anne has suggested that her eventual replacement
be an expert in teaching and facilitating writing across the
curriculum. 

\item We should seek to endow as many of these new positions as
possible.  We will also continue to seek to endowment to support
existing positions and programs.  In the current budget climate, we do
not recommend adding any faculty positions unless they are supported
by endowment or there are already on campus sufficient students (and
hence tuition revenue) to support the new faculty's salary while also
making progress on salary equity. 

\item Regularizing visitors in arts and design is the number two
priority in the recently completed strategic plan for the arts and
design area.  (The first priority is improved studio teaching space;
the third concerns plans for the gallery.)

\item John Cooper has prepared a detailed list of needs associated
with music as part of the arts and design plan. 

\item We will continue to expand opportunities for students to work on
writing across the curriculum and will seek additional training and
support for faculty who wish to teach writing-focused classed.
Academic affairs tentatively recommends seeking a lectureship in
writing as soon as is feasible. 

%\item See also the list of questions below in Section
%\ref{questions}. 

\end{enumerate}



\section{Lower priorities}

Of this next set of priorities I propose dividing them into two
tiers.  The second tire strikes me as ideas for which there is
considerable support and/or need.  If one of these positions were
suddenly fully endowed, we'd be psyched. My sense is that ideas in the
third tier are less broadly supported and/or understood at present.\\ 

\noindent {\bf Second Tier}
\begin{enumerate}
\setlength{\itemsep}{-1mm}
    \item Geography
    \item Health Sciences/Public Health
    \item Religion
\end{enumerate}

\noindent {\bf Third Tier}
\begin{enumerate}
\setlength{\itemsep}{-1mm}
    \item Art History
    \item Computer Sciences/Mathematics
    \item Media Studies
    \item U.S.~History
    \item Writing (as a full-time hire)
\end{enumerate}


\section{Next Steps}
\label{questions}

\begin{enumerate}

\item This document will be refined by AAC, with particular focus on
visiting classes and lectureships.

\item Most of the proposed new positions need considerable shaping and
articulation.  Interested faculty should work over the next several
terms to prepare brief narratives describing and justifying potential
new hires.  This include second and third tier positions in addition
to top tier positions.

\end{enumerate}


\section{Open Questions}

There are (at least!) several questions that are still open and which
I want to acknowledge as such.  These include:

\begin{enumerate}

\item What is the role of taxidermy in the curriculum?

\item What is meant by Asian Studies?  My sense is that people mean
very different things by this.  

\item What should an additional hire in arts be?  If we grow to 300,
it seems certain to me that we'll need an additional arts and design
faculty member.

\end{enumerate}



\section{Lectureships and Visiting Courses}
\label{lectureships}

We should arrange for lectureships and other, more regularized
visiting classes in the following areas:

\begin{enumerate}
\setlength{\itemsep}{-1mm}
    \item Business (via Organizational Stewardship)
    \item Music
    \item Maintain lectureship in Ecology/Natural History and
    Ornithology 
    \item Photography
    \item Writing
    \item Spanish language instruction
    \item Taxidermy
\end{enumerate}

Additionally, we will have ongoing visiting class needs in the
following:
\begin{enumerate}
\setlength{\itemsep}{-1mm}
    \item Dance
    \item Education
    \item Mathematics and Physics 
    \item Multimedia and Digital Design
    \item Theater
    \item U.S.~History
    \item Writing and Literature
\end{enumerate}

Remaining visiting classes should be used opportunistically to enrich
and enhance the curriculum.  Areas of particular interest include, but
are most certainly not limited to:  Asian, African, and Middle East
studies; astronomy; meteorology; bryophytes and lichens. 



\appendix

\section{Numerical observations and thoughts on visiting classes}
\label{numbers}

\begin{enumerate}
    \item We currently have 25.51 faculty FTE and a student-faculty
    ratio of almost exactly 10.

    \item This (and subsequent) student-faculty ratios are based on
    treating fractional appointments as fractions.  (E.g., I count
    Chris as 0.5 because he technically has a half-time appointment.)
    However, the standard way that colleges report these numbers is
    different.  The standard reporting will make our student faculty
    ratio appear lower.  

    \item To maintain a 10:1 student-faculty ratio we will need to add
    at least 4.5 new faculty.

    %\item This assumes that the fraction of courses taught by visitors
    %will stay the same.  (The fraction is currently around 20\%.)

    \item Over the past six terms (F03--S05), 25\% of our classes have
    been taught by non-permanent faculty.  This percentage is quite
    constant; it ranged from 22 to 28 over the six-term period.  This
    statistic does not include tutorials.  

    \item I have not been able to find reliable data on the
     percentage of classes taught by adjunct faculty at other
     colleges and universities.  The reason for this is that it's not
     clear how other colleges calculate this.  (In particular, it's
     unclear how they count graduate student instructors.)  So far as
     I can tell, most don't report this statistic at all.  The best
     overview of this question that I found is Ernst Benjamin, How
     Over-Reliance on Contingent Appointments Diminishes Faculty
     Involvement in Student Learning\footnote{
     {\url{http://www.aacu-edu.org/peerreview/pr-fa02/pr-fa02feature1.cfm}},
     accessed 02.13.05.} 

    \item Here are some somewhat fragmentary data.  At baccalaureate
     (i.e., non-doctoral) institutions, around 25\%
     introductory calculus classes are taught by a non-tenure-track
     faculty member\footnote{{\em ibid.}}  Nationally, the
     percentage of classes offered by non-permanent faculty in history
     departments is around 35\%\footnote{
     {\url{http://www.theaha.org/perspectives/issues/2000/0010/pt_survey.htm}},
     accessed 2.13.05.}.  
     The percentage at Iowa
     State\footnote{
     {\url{http://www.iastate.edu/Inside/04/0521/senate.shtml}},
     accessed 2.13.05.} is 24.4 across all disciplines, although I'm
     not sure how they 
     determined this statistic.  In some departments at Iowa State,
     the percentage is much higher---e.g. 39.5\% in Ecology, evolution
     and organismal biology and 33\% in English. 

     \item What is clear is that nationally the trend is that the use
     of adjunct instructors is on the rise.  Several studies have
     found that the use of adjunct instructors has roughly doubled from
     1970 to the mid 1990s. 

     \item Although our percentage of classes taught by visiting
     faculty is well below the national average, many professional
     societies suggest that the ideal percentage of classes taught by
     visitors is lower than what ours currently is.  For example, 
     the American Historical Association\footnote{
     {\url{http://www.historians.org/press/2003_05_05_Council_Parttime.htm}},
     accessed 2.13.05.} recommends that at four-year institutions
     between 10\% and 20\% of classes be offered by visitors.  The
     American Association of University Professors\footnote{
     {\url{http://www.aaup.org/Issues/part-time/Ptguide.htm}},
     accessed 2.13.05.} recommends that no more than 15\% of a college
     or university's instructional load be carried by non-tenure-track
     faculty. 


    \item If we stay at 10:1 and still have 25\% of our classes taught
    by visitors, we will need to have between 17 and 23 visiting
    courses every fall.  This strikes me as a number that is close to
    unmanagably large.  

    \item If we decrease the fraction of courses taught by visitors I
    estimate that we could very easily add up to 7.5 faculty.

    \item For example, with 33 faculty (and a 9.1:1 ratio) we would
    still need around between 13 and 18 visiting classes each fall.
    This is essentially the same number of visiting classes we
    currently need each fall.  

%    \item An analysis of the financial viability of different
%    enrollment sizes is ongoing.  Andy Griffiths is leading this
%    effort.
  
    \item If we have 300 students I expect that some of the largest
    classes will grow by up to $20$\%. This means that, for example,
    calculus would be around 30 students instead of 25.  


\end{enumerate}

\section{Notes on Possible Science Hires}

At the science resource area meeting of 02.09.05, we spoke at some
length about possible hires and curricular directions.  All ES faculty
members were in attendance, except for Suzanne Morse, who is currently
in Mexico.  The following were key points of our discussion:
\begin{enumerate}

\item There was a clear consensus that our top priority for an
additional hire is in geology.  We feel this will fill in gaps in our
curriculum and help form bridges to other areas of the college.  This
hire could also give Don and Dave some much needed help in physical
sciences and quantitative reasoning.

\item There wasn't a clear consensus on our next highest priority,
although the group was leaning toward a position in health.

\item There is some confusion about the future role of GIS in the
curriculum.

\item There was consensus that we wish to continue affiliation
with The Jackson Lab and MDIBL in genetics and medical research.
However, we do not see ourselves as a pre-med school.  While students
can attend here at go on to medical or veterinary school, we should
not make this a advertising focus of the college, nor should we
structure our curriculum around preparing students for medical school.  

\item We did not discuss Beech Hill Farm and Food Systems.

\end{enumerate}

\subsection{Rough Descriptions for Possible Science Hires}

These are very, very rough, and are designed to give a flavor of what
we're thinking about in these positions.

{\bf Geologist:}  We seek a broadly-trained earth scientist/geologist
with wide-ranging intellectual interests.  Possible areas of expertise
include: Hydrology/Hydro-geology, Geology of Rivers and rivers
restoration, Water Quality and/or water use issues, Desertification,
Oceanography, Erosion.  We are particularly interested in candidates
who have studied the interactions between human activity and the
physical environment, (e.g. erosion, climate change, desertification).
We are also interested in someone who has experience or expertise in a
number of policy and related areas.  For example, the following
experiences are highly desirable: Work on government or
non-governmental restoration projects; familiarity with US or
international regulatory agencies; experience conveying environmental
issues to the public; experience with community-based research
projects.  In addition to courses in appropriate areas of geology, we
expect that the new hire will be able to teach some of the following
courses:  Intro to Chemistry, Physical Chemistry, Statistics,
Scientific Programming, Introduction to Physics.  This will 


{\bf Health Sciences:} We are looking for someone who could offer
courses in some of the following areas: Community Health, Public
Health, Nutrition, Epidemiology, International Public Health,
Environmental Health, Toxicology, Human Genetics, Biostatistics,
Public Health Research Methods and Data Analysis.  We do not see this
as a position in anatomy or one geared toward medical students.  

{\bf Computer Sciences/Mathematics:}  Interested in a broadly trained
computer scientist who can teach classes in the following areas:
Introduction to Computer Science; Introduction to Programming;
Intermediate Programming and Data Structures; other areas of
mathematics.  Other areas of interest include: computer-human
interactions; artificial intelligence; applications of computational
approaches to biology, economics, physics, chemistry; scientific
programming and numerical analysis; technology and society.  This is
not a position in The Web or using particular programs.  The emphasis
is on teaching students to write their own programs.  This position
will also provide additional mathematics classes.  

\end{document}
