\documentclass[12pt]{article}
\oddsidemargin=0.0in
\evensidemargin=0.0in
\textwidth=6.5in
%\topmargin=-.75in
%\textheight=9in
\usepackage{url}
 
\begin{document}
\pagestyle{empty}
 
\begin{center}
{\large {\bf Thermodynamics}}\\
\medskip
{\large {\bf Homework Two}}\\
\medskip
{ {\bf Due Friday, April 8, 2016}}\\
\end{center}

\medskip

\indent This assignment is likely longer and more challenging than the
first one.  

\begin{enumerate}
\setlength{\itemsep}{1mm}
  \item 1.31
  \item 1.35
  \item 1.37
  \item 1.41
  \item 1.45 Optional.  Recommended for Calc III
    alumni/ae. It's not a difficult calculation.
  \item 1.47 or 1.48. Do whichever looks more interesting. Or do both
    if they both look really interesting.
  \item 1.50 This might get delayed until next week.  It'll depend on
    how far we get in class on Wednesday.
  \item 1.54 Optional, but recommended.  Since we're interested in
    just getting rough estimates, we can ignore the difference between
    energy and enthalpy for this problem. 
\end{enumerate}

%  \item Tastybites (\url{http://www.tastybite.com}) are tasty pouches
%    of Indian or pan-asian food.  You are camping and your dinner will
%    involve a tastybite entr\'ee.  You boil a litre of water, pour the
%    water into a pot, put the pouch in the pot with the water, and put
%    a lid on the pot.  You then proceed to make rice.  Estimate the
%    temperature of food in the tastybite pouch when the rice is done.
%    Be sure to state any simplifying assumptions you make.  (This is
%    meant to be a simple specific heat problem.  Don't over think it.)



\end{document}
