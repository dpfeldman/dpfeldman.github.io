
\documentclass[12pt]{article}
\oddsidemargin=0.0in
\evensidemargin=0.0in
\textwidth=6.5in
\topmargin=-.45in
\textheight=8.4in
%\usepackage{doublespace}
 
\begin{document}
\pagestyle{empty}
 
\begin{center}
{\large {\bf Thermodynamics}}\\
\medskip
{\large {\bf Homework Seven}}\\
\medskip
{ {\bf Due Friday, May 13, 2016}}\\
\end{center}

\noindent This assignment will be a bit longer than usual, since it
covers 1.5 weeks instead of one.  However, I think these problems will
be much less mathematically demanding than those on the last problem
set, and will involve a good bit more interesting physics.

\begin{enumerate}

  \item Problem 4.14.  Heat Pumps!
  \item Problem 4.18  Optional.  This problem involves a modest amount
    of algebra and could be a good review of the physics of adiabatic
    and isothermal processes. 
  \item Problem 4.20  Optional.  This problem involves a less modest
    amount of algebra.  I'm not sure the process of doing this problem
    will lead to deep understanding, but the final result---the
    efficiency of a Diesel engine---is a useful and important result,
    although it is not a simple formula.  
  \item Problem 4.22
  \item Problem 4.34 Optional. Looks at what happens if we relax the
    assumption that $H_1$ = $H_2$. 
%  \item Problem 4.24  
  \item Problem 4.26
  \item Problem 3.37 Optional. This problem uses the chemical
    potential to derive the exponential atmosphere.  You can do part
    (b) without doing part (a) if you want.
  \item Problem 5.4
  \item Problem 5.6
%  \item Problem 5.11
%  \item Problem 5.20

\end{enumerate}


\end{document}
