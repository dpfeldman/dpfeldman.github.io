
\documentclass[12pt]{article}
\oddsidemargin=0.0in
\evensidemargin=0.0in
\textwidth=6.5in
%\topmargin=-.75in
\textheight=8.4in
%\usepackage{doublespace}
 
\begin{document}
\pagestyle{empty}
 
\begin{center}
{\large {\bf Thermodynamics}}\\
\medskip
{\large {\bf Homework Six}}\\
\medskip
{ {\bf Due Friday Mary 5, 2016}}\\
\end{center}


\begin{enumerate}
%  \item Problem 3.37.  Optional.  This problem uses the chemical
%    potential to derive a formula for the exponential atmosphere. 
  \item Problem 1.50
  \item Problem 1.51
  \item Problem 4.2
  \item Problem 4.3
%  \item Problem 4.5.  Should be a fairly straightforward but
%    multi-step calculation, using ideas from chapter 1.  
  \item Problem 4.6.  This is a challenging, multi-step problem.  Do
    the best with it that you can.  
  \item Problem 4.8
  \item Problem 4.13.  The last part of the problem asks for a
    numerical example.  Here is one way to put numbers to this
    problem.  Suppose that you want the temperature inside your house
    to be $25$.  When it is $30$ degrees outside, your air conditioner
    costs \$1/hr to run.  If it was $35$ outside and you wanted to it
    be $25$ inside, how much would your air conditioner cost you per hour?
%  \item Problem 4.14.  Heat pumps!
\end{enumerate}


\end{document}
