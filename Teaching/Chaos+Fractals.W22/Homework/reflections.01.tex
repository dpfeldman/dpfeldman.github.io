\documentclass[12pt]{article}
\oddsidemargin=0.0in
\evensidemargin=0.0in
\textwidth=6.5in
\topmargin=-0.55in
\textheight=9in
%\usepackage{doublespace}
\usepackage{hyperref}

\hypersetup{
    colorlinks=true,
    linkcolor=magenta,
    filecolor=magenta,      
    urlcolor=blue,
    %pdftitle={Overleaf Example},
    %pdfpagemode=FullScreen,
    }

\begin{document}
\pagestyle{empty}
 
\begin{center}
{\LARGE {\bf First Reflection Assignment}}\\
\bigskip
{\Large {\bf Chaos and Fractals}}\\
\bigskip
{\Large {\bf College of the Atlantic}}\\
\bigskip
{{\bf Target Due Date: Monday, 31 January}}\\
\end{center}
\medskip


\noindent {\bf Goal.} The goal of this assignment is for you to spend
time reflecting on and digesting some of the themes of this course.
Explore some ideas from the class that have captured your imagination.
Your piece should show evidence of sustained inquiry.  Find something
that interests you or perplexes you and explore.  Push your ideas and
(gently) push yourself.  Ideally, your piece should be emotionally and
intellectually satisfying to you.  Choose a question, word, phrase,
issue, image, equation, person, quotation, or idea from the readings
or discussion and dig deeply.  \\

%Your paper should reflect both deep analytic thought and intellectual
%risk-taking. Find something that really interests you to
%explore. Push the idea as far as you can and push yourself. Make a
%discovery and ensure that your reader will not be bored by what you
%see and say. Defend your perspective. Above all, the response paper
%should be intellectually and emotionally satisfying. In approaching
%the response paper, plan to sit down at your computer or journal and
%address something you can really develop passion and concern 
%about. 

\noindent {\bf Format.} I suspect many of you will write a paper that
is an essay or is generally essay-like.  If so, your paper should be
around two-three pages in length.  However, there are many other
options: 
\begin{itemize}
\item You could write a short story, a lyric essay, a poem.
\item You could create a drawing, a cartoon, a painting.
\item You could collaborate with someone else in the class and record
  a dialog, or jointly create something.
\end{itemize}
If you choose a non-essay option, your piece should be roughly
equivalent to a two-three page essay.  (I honestly don't know what
that means.)\\

\noindent {\bf What to Explore.} There are many possibilities! We've
talked quite a bit in this class about different notions of randomness
and order. This week in discussion section we'll look at the nature of
scientific change. Maybe there's a word or phrase or image that is
interesting to you that could be a good jumping off point. If you're
unsure of what to explore (or what formats to use for your
exploration), let me know and we can brainstorm together.\\
  


\end{document}
