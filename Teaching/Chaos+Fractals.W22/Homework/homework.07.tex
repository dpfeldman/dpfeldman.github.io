\documentclass[11pt]{article}
\oddsidemargin=0.0in
\evensidemargin=0.0in
\textwidth=6.5in
\topmargin=-0.75in
\textheight=9.8in
%\usepackage{doublespace}
\usepackage{hyperref}

\hypersetup{
    colorlinks=true,
    linkcolor=magenta,
    filecolor=magenta,      
    urlcolor=blue,
    %pdftitle={Overleaf Example},
    %pdfpagemode=FullScreen,
    }

\begin{document}
\pagestyle{empty}
 
\begin{center}
{\LARGE {\bf Homework Seven}}\\
\smallskip
{\Large {\bf Chaos and Fractals}}\\
\smallskip
{\Large {\bf College of the Atlantic}}\\
\smallskip
{ {\bf Due Friday 18 February, 2022\footnote{If you need more time,
      that's ok. Just let me know. Thanks!}}}\\  
\end{center}
%\medskip


%{\bf {\LARGE This problem set is not complete. I might add a few
%    additional problems.}} \\

\noindent There are two parts to this week's homework.\\


\noindent {\bf Part 1: WeBWorK}.
Do Homework 07 which you will find
on your
\href{https://webwork.runestone.academy/webwork2/coa-feldman-es1026i-winter-2022}{WeBWorK
  page}.  I recommend doing the WeBWorK part of the 
homework first.  This will enable you to benefit from WeBWorK's
instant, if not necessarily friendly, feedback before you do part two.\\


\noindent {\bf Part 2: Problems from the Textbook}.  Here are some
instructions for how to submit this part of the assignment.
\begin{itemize}
\setlength{\itemsep}{0mm}
\item Do the problems by hand using pencil (or pen) and paper.
  There is no need to type of this assignment.
\item Make a pdf scan of your work using genius scan or some
  similar scanning app.  Please make the homework into a single
  pdf, not multiple pdfs.
\item Submit the assignment on google classroom.
  %Please don't email it to me.  (Between my two classes I will be
  %receiving around 45 assignments a week.  Keeping track of them all
  %in email is challenging.)
\item {\bf If you want, do these problems in pairs.} %Most of these
%  problems are exploratory in nature, and it might be fun to have a
  %  co-explorer.
  If you do work in pairs,
  please submit only one set of solutions for the two of you.  If you
  work with a friend and your friend submits the work, please submit
  an empty assignment and add a comment letting me know who you worked
  with.  Thanks.
%\item Do problem 40 in pairs. Work with someone else in the class
%  and hand in only one solution for the two of you.
%\item If you want, you can do all the problems in pairs and hand
%  in only one set of solutions.\\
\end{itemize}

\noindent Here are some ``textbook'' problems, which aren't actually
from the textbook.

\begin{enumerate}
\item Consider the following
  complex numbers:
  \begin{equation}
    z_1 = -4 -2i \;,\;\;\; z_2 = 3i \;,\;\;\; z_3 = 2 + 0.5i \;,
    \;\;\;   z_4 = -2 +0.5i \;,
  \end{equation}
  \begin{equation}
  z_5 = -2i \;,\;\;\; z_6 = 2 + 0.5i \;, \;\;\;    z_7 = 4 -2i
  \;,\;\;\; z_8 = -4 -2i \;.
  \end{equation}
    \begin{enumerate}
      \item Plot the above numbers\footnote{Yes, I know
        that there are two numbers that are there twice.} on the complex plane. 
      \item Connect the dots.  The pattern should look familiar.
    \end{enumerate}

  \item Consider the function $f(z) = iz$.
    \begin{enumerate}
      \setlength{\itemsep}{0mm}
    \item Determine first four iterates of $z_0 = 3$.
    \item Determine first four iterates of $z_0 = 2i$.
    \item Plot the iterates for each of the seeds in the complex
      plane.
    \item How would you describe the behavior of the orbits?
    \end{enumerate}
      
\end{enumerate}

\noindent And here are some textbook problems from the textbook,
Chapter 21:  
\begin{enumerate}
  \setlength{\itemsep}{0mm}
\item 21.1
\item 21.2
\item 21.3
\item 21.4
\item 21.5
\item Optional: 21.8, 21.9, 21.20
\end{enumerate}



\end{document}
