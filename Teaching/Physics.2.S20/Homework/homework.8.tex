\documentclass[12pt]{article}
\oddsidemargin=0.0in
\evensidemargin=0.0in
\textwidth=6.5in
%\topmargin=-.75in
%\textheight=9in
\usepackage{url}
\usepackage{hyperref}
 
\begin{document}
\pagestyle{empty}
 
\begin{center}
{\large {\bf Calculus II}}\\
\medskip
{\large {\bf Homework Eight}}\\
\medskip
{ {\bf Due Friday March 1, 2019}}\\
\end{center}

\noindent {\bf Link to Webwork:
  \url{https://courses1.webwork.maa.org/webwork2/COA-ESL024}} \\ 



\noindent {\bf Chapter 8.7:}  
\begin{itemize}
\setlength{\itemsep}{0mm}
\item WeBWorK Assignment: 07Chapter8.7and8.8
\item Textbook Problems:
  \begin{enumerate}
  \setlength{\itemsep}{-1mm}
    \item None
  \end{enumerate}
\end{itemize}

\noindent {\bf Normal Distributions}

\begin{enumerate}
\setlength{\itemsep}{-1mm}
\item The height of giraffes is distributed according to a normal
  distribution with a mean of $5.2$ and a standard deviation of
  $0.6$.  
\begin{enumerate}
\setlength{\itemsep}{-1mm}
\item What fraction of giraffes are less than 4 meters tall?
\item What fraction of giraffes are between 5 and 6 meters tall?
\item What fraction of giraffes are more than 5.5 meters tall?
\end{enumerate}
Answer these questions two ways:
\begin{itemize}
\setlength{\itemsep}{-1mm}
\item Using WolframAlpha to evaluate the integrals.  You do not need
  to show printouts from wolfram alpha, but you should write down the
  integrals that you are asking wolfram alpha to solve for you.
\item Converting to z and using a z-table. See, e.g.,
  \url{www.stat.ufl.edu/~athienit/Tables/Ztable.pdf}.  Briefly explain
  how you used the z table to answer each of the questions. 
\end{itemize}

\item Darron Collins is interested in the heights of COA students
  compared to Hampshire students.  A careful study reveals that COA
  students have an average height of 63 inches and a standard deviation of 4
  inches.  Darron then sends a team of researchers on a trip to Massachusetts
  to measure the heights of some Hampshire students.  The RA team
  manages to convince 25 Hampshire students to be measured.  The mean
  of these 25 Hampshire students is 64 inches. 
  %The standard deviation of this sample of Hampshire students is 3.
\begin{enumerate}
\setlength{\itemsep}{-1mm}
  \item If you sampled 25 COA students, how would the mean of that
    sample be distributed? 
  \item Given this distribution, what is the z-score for the measured
    average height of Hampshire students?
  \item How likely is it that sampling 25 COA students would lead to a
    mean equal to more more extreme\footnote{Note that unlike in the
      unicorn example, z is now positive.  So you'll have to think a
      little about how to interpret ``more extreme than'' for this case.} than the mean you found?
  \item Given this experiment do you think it is likely that the
    average heights of Hampshire and and COA students are the 
    same? 
\end{enumerate}

\item Repeat the above question, but suppose that the RAs measured 100
  Hampshire students and found an average height of 64 inches.  

\end{enumerate}



\noindent {\bf Chapter 9.1:}  
\begin{itemize}
\setlength{\itemsep}{0mm}
\item WeBWorK Assignment: 07Chapter9.1
\item Textbook Problems:
  \begin{enumerate}
  \setlength{\itemsep}{-1mm}
    \item 2
    \item 4
    \item 8
    \item 10
    \item 20-28 (even only)
  \end{enumerate}
\end{itemize}




\end{document}
