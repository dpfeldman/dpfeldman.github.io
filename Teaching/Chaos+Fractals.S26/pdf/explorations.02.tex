\documentclass[12pt]{article}
\oddsidemargin=0.0in
\evensidemargin=0.0in
\textwidth=6.5in
\topmargin=-0.55in
\textheight=9in
%\usepackage{doublespace}
\usepackage{hyperref}

\hypersetup{
    colorlinks=true,
    linkcolor=magenta,
    filecolor=magenta,      
    urlcolor=blue,
    %pdftitle={Overleaf Example},
    %pdfpagemode=FullScreen,
    }

\begin{document}
\pagestyle{empty}
 
\begin{center}
{\LARGE {\bf Bifurcation Diagram Exploration!}}\\
\bigskip
{\Large {\bf Chaos and Fractals}}\\
\bigskip
{\Large {\bf College of the Atlantic}}\\
\bigskip
    {{\bf Please do these before class on Friday, 29 September}}\\
\end{center}
\medskip

\noindent General Instructions

\begin{itemize}
\setlength{\itemsep}{0mm}
\item Do this with others if you like. It might be more fun that
  way.
\item We'll go over this in class. There's nothing to hand in. 
\item Don't spend more than 10-15 minutes on these exploration (unless
  you want to).
  \item Use this web page for iterating:
    \href{https://s3.amazonaws.com/complexityexplorer/DynamicsAndChaos/Programs/bifurcation.html}{\url{https://s3.amazonaws.com/complexityexplorer/DynamicsAndChaos/Programs/bifurcation.html}}.    
\end{itemize}


\noindent In this exploration you will investigate the period-doubling
route to chaos on the bifurcation diagram.  In the bifurcation diagram
we see that the behavior of the orbits shifts from period one (a fixed
point) to period two at $r=3.0$. This is a \emph{bifurcation}---a
sudden change in behavior.
%Let's call the value at which this bifurcation occurs $r_1$, since it
%is the value at which the first bifurcation occurs.
By zooming in on the bifurcation diagram, locate
the $r$ values at which subsequent bifurcations occur. Try to determine
these $r$ values to several decimal places, although this may not be
possible.  

\begin{enumerate}

  \item Find the $r$ value at which the orbits shift from period $2$
    to period $4$.
  \item Find the $r$ value at which the orbits shift from period $4$
    to period $8$.
  \item Find the $r$ value at which the orbits shift from period $8$
    to period $16$.  
\end{enumerate}


\end{document}
