\documentclass[12pt]{article}
\oddsidemargin=0.0in
\evensidemargin=0.0in
\textwidth=6.5in
\topmargin=-0.35in
\textheight=9in
\usepackage[utf8]{inputenc}
\usepackage[russian]{babel}
\usepackage[normalem]{ulem}
%\usepackage{doublespace}
 
\begin{document}
\pagestyle{empty}
 
\begin{center}
{\LARGE {\bf Homework Five}}\\
\medskip
{\Large {\bf Thermodynamics}}\\
\medskip
{\Large {\bf College of the Atlantic}}\\
\medskip
{ {\bf Due Friday, Feb 7, 2025}}\\  
\end{center}
\medskip


\noindent All problems are from the textbook, unless otherwise stated. 


\begin{enumerate}
  \setlength{\itemsep}{4mm}

\item Deriving a useful approximation.
  \begin{enumerate}
  \item What is the derivative of $\ln(1+x)$?
  \item Evaluate this derivative for $x=0$.
  \item Figure out the equation of the line tangent to $\ln(1+x)$ at
    the point $x=1$. You should find that the equation of the tangent
    line is simply $y=x$.
  \item You have thus derived the approximation we've used in class
    repeatedly over the last several days:
    \begin{equation}
      \ln(1+x) \, \approx x \;,
      \label{eq:approx}
    \end{equation}
    which is valid for $|x|\ll 1$.  
    \item Check the accuracy of the approximation in
      Eq.~(\ref{eq:approx}) for $x=0.1$, $x=0.01$, and
      $x=0.001$. I.e., for each value of $x$ evaluate the left-hand
      side of Eq.~(\ref{eq:approx}) using a calculator, and compare it
      two the right-hand side.
  \end{enumerate}

  
\item Suppose you flip $1000$ coins.
  \begin{enumerate}
    \item Write down an expression for the multiplity of the
      macrostate for $500$ heads and $500$ tails.
    \item Write down an expression for $\Omega_{\rm all}$, the total
      number of microstates. (I.e., the total number outcomes that can
      occur if you flip $1000$ coins.
    \item Determine the probability of the macrostate with $500$ heads
      and $500$ tails. Do so by using Sterling's approximation:
      \begin{equation}
        N! \approx N^Ne^{-N}\sqrt{2\pi N} \;.
      \end{equation}
  \end{enumerate}

\item 2.21 (Use WolframAlpha or desmos or whatever you're used to
  using to make plots.)

\item 2.26

\item {\bf Optional:} 2.17. In this problem you'll determine an
  expression for the multiplicity of an Einstin solid for $q <<
  N$. Good practice using Sterling's approximation and the Taylor
  expansion for the natural log, if that's your thing.

  
\end{enumerate}






\end{document}
