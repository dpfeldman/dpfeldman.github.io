\documentclass[12pt]{article}
\oddsidemargin=-0.20in
\evensidemargin=-0.20in
\textwidth=6.5in
\topmargin=-.28in
\textheight=9.2in
%\usepackage{doublespace}
 
\begin{document}
\pagestyle{empty}
 
\begin{center}
{\large {\bf Introduction to Computer Science}}\\
\medskip
{\large {\bf Homework Five}}\\
\medskip
{ {\bf Due Sunday February 6, 2011}}\\ 
\end{center}

%\hspace{2mm}\\

\noindent General Instructions

\begin{itemize}
\setlength{\itemsep}{-1mm}
  \item Please email me the programs as a .py attachment.
  \item Name your programs with a helpful name.  Include your name
    somewhere in the file name.
  \item Your program should be fully commented.  Be sure to include
    your name and the date and an overall description of what the
    program does, in addition to any other comments that are needed. \\
\end{itemize}



\noindent This is likely a more annoying assignment than the ones
we've done thus far.  Give yourself some time to work on it, and if
you  have questions or need help, be sure to ask.\\

\noindent   Your task is to write two programs.  One program will
encrypt a plaintext file.  The other program will read in the
encrypted file and decode it.  There is a third text file that you
should use as a key. \\

\noindent   The encryption program should:

\begin{itemize}
    \item Read in a plaintext file.  The user should be able to
      specify this file.

    \item Encrypt it using another file as a key.  The user should
      enter the name of the key file.

    \item The encrypted file---i.e. the ciphertext---should be written
      to a third file.  Again, the user should specify the name for
      the ciphertext file.  

\end{itemize}


\noindent The decryption program should do the opposite.  It should
read in the cipher text, decode it using the key, and then print out a
plaintext file. \\

\noindent For the encryption scheme I would recommend the following.
Convert both the plain text and the key to unicode.  Add these
together, character by character.  Then take these numbers and convert
back to characters.  (The characters will look odd, since the unicode
numbers will be mostly beyond the regular alphabet.  But they'll still
be readable in a plain text file.)  \\

%\hspace{2mm}\\

\noindent {\bf Alternative Assignment.}  I am certainly open to other
assignments if this one doesn't sound exciting to you.  The main thing
I want you to practice is reading in information from a file that may
not be in the exact format that you want, doing something with that
information, and then writing that info to another file.  There are
lots of statistics/data analysis things you could do.  You could also
have some fun transforming text.  If you want to do an alternative
assignment, please clear it with me first.  \\

\noindent {\bf Optional Problems.}  I recommend doing these if you
have time. 
\begin{itemize}
\setlength{\itemsep}{-1mm}
  \item Exercise 4, page 162 of Zelle
  \item Exercise 10, page 163 of Zelle
\end{itemize}


\end{document}
