\documentclass[12pt]{article}
\oddsidemargin=0.0in
\evensidemargin=0.0in
\textwidth=6.5in
%\topmargin=-.75in
%\textheight=9in
%\usepackage{doublespace}
 
\begin{document}
\pagestyle{empty}
 
\begin{center}
{\large {\bf Introduction to Computer Science}}\\
\medskip
{\large {\bf Homework Two}}\\
\medskip
{ {\bf Due Sunday January 16, 2011}}\\ 
\end{center}

\hspace{2mm}\\

\noindent General Instructions


\begin{itemize}
\setlength{\itemsep}{-1mm}
  \item Email me the program as a .py attachment.
  \item Name your programs with a helpful name.  Include your name
    somewhere in the file name.
  \item Your program should be fully commented.  Be sure to include
    your name and the date and an overall description of what the
    program does, in addition to any other comments that are needed. 
\end{itemize}


\begin{enumerate}

\item Write a program that does the following:
\begin{enumerate}
\setlength{\itemsep}{-1mm}
  \item Greets the use and introduces the program.
  \item Prompts the user for an amount of money in Euros.
  \item Prints the corresponding amount in US Dollars.
  \item Prints a farewell message.

\end{enumerate}

If you want to do some other currency conversion, or any other
conversion, that's ok.\\

\item Write a program that calculates the future value of an
  investment, given a percentage growth rate and the initial amount of
  the investment.  The program should:
\begin{enumerate}
\setlength{\itemsep}{-1mm}
  \item Prompt the user to input an amount of money.
  \item Prompt the user to enter the annual percentage growth of the
    investment. 
  \item Prompt the user to enter a number of years $n$.
  \item Prints the value of the investment in $n$ years.
  \item Print a farewell message.\\

\end{enumerate}


\item Write a program that calculates the area of pizzas.  Use the 
  math module to get the value of $\pi$.  The program should:
\begin{enumerate}
\setlength{\itemsep}{-1mm}
  \item Prompt the user to input the radius $r$ of the pizza.
  \item Prints the area of the pizza.
  \item Prints the area of a pizza with a pizza with radius $r+2$.  
  \item Prints the area of a pizza with a pizza with radius $r-2$.  
  \item Print a farewell message. \\
\end{enumerate}


\item ({\bf Optional})  Suppose you want to save for a large purchase
  a house or college for your kids or something.  Instead of
  depositing a lump sum of money into the bank, instead you deposit a
  fixed amount of money $M$ every year.  The money in the bank earns
  $r$ percent interest yearly.  
\begin{enumerate}
\setlength{\itemsep}{-1mm}
  \item Write a program that, given the yearly deposit amount $M$, the
    interest rate $r$, computes the amount of money in the bank $N$
    years from now.   
  \item Use your program to figure out how much money you would need
    to deposit every year if you want to have \$20,000 in 20 years.
    Assume that the interest rate is $3$\%. \\
\end{enumerate}


\item ({\bf Optional})  Write a program that iterates the logistic
  equation, $f(x) = 4x(1-x)$.  
\begin{enumerate}
\setlength{\itemsep}{-1mm}
  \item Have the user enter:
\begin{enumerate}
\setlength{\itemsep}{-1mm}
  \item Two different initial values.  Each initial value must be
    between $0$ and $1$.
  \item The number $n$ of iterates to calculate.
\end{enumerate}
  \item The program should then print out $n$ iterates for each of the
    initial conditions.
\end{enumerate}




\end{enumerate}


 
\end{document}
