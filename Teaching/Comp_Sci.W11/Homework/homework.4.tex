\documentclass[12pt]{article}
\oddsidemargin=-0.20in
\evensidemargin=-0.20in
\textwidth=6.5in
\topmargin=-.25in
\textheight=9in
%\usepackage{doublespace}
 
\begin{document}
\pagestyle{empty}
 
\begin{center}
{\large {\bf Introduction to Computer Science}}\\
\medskip
{\large {\bf Homework Four}}\\
\medskip
{ {\bf Due Sunday January 30, 2011}}\\ 
\end{center}

\hspace{2mm}\\

\noindent General Instructions

\begin{itemize}
\setlength{\itemsep}{-1mm}
  \item Email me the program as a .py attachment.
  \item Name your programs with a helpful name.  Include your name
    somewhere in the file name.
  \item Your program should be fully commented.  Be sure to include
    your name and the date and an overall description of what the
    program does, in addition to any other comments that are needed. 
  \item My plan is to post the random graphics program on the website,
    since I thought it might fun to look at each other's creations.
    If you don't want your program posted, just let me know. \\
\end{itemize}


\begin{enumerate}

\item This is a more complex assignment than the ones we've done thus
  far.  Give yourself some time to implement it, plan out the
  structure of the program, and be sure to use ample comments so it is
  clear to read.

  Your task is to write a program the uses Zelle's graphics package
  and some random number features to make an interesting or
  aesthetically appealing picture using random shapes and colors.  You
  can use circles or triangles or something else or a combination.
  I would recommend setting colors randomly as in the example program
  I used in class.  If you restrict the colors so as to not go from
  $0$ to $255$ you can produce some interesting effects.

  Your program needs to incorporate the following
\begin{itemize}
    \item Should use two GraphWin windows.  One window will contain
      the picture that gets created.  The other window should contain
      some information about the program and some boxes into which the
      user can enter some choices, perhaps the total number of shapes
      or the size range of the shapes or something.

    \item Use the setCoords option for the main graphics window so
      that the program works for different sizes of the window.

    \item Use the getMouse() method at least once.  Most likely you
      will want to have the user click when he or she is done entering
      information and is ready to start drawing.

    \item Use the randrange function at least once to randomize some
      aspect of the pictures.  (You'll probably want to use it
      multiple times.)
\end{itemize}
After the assignment I've included some tips and advice for writing
this program. \\


\item Write a program that produces the average of a bunch of numbers.
  The program should print a welcome message and then ask the user how
  many numbers he or she wants to average.  The user should then enter
  the numbers.  Then the program should then output the average and
  print a farewell message.  ({\bf Optional:} Also have the program
  calculate and output the standard deviation of the numbers.) 


\item {\bf Optional:}  The getMouse() method returns the point on the
  window where the mouse was clicked.  This information can then be
  used in the program.  To try this out, write a program that makes a
  window, has a user click on two points, and then draws a rectangle.
  See section 4.7.2 of Zelle for details.  Alternatively, you could
  have the user draw a circle by specifying the radius and then a
  point on the circle.  (This one is a little more challenging.) 


\end{enumerate}


\noindent {\bf Some Advice for the Random Graphics Program}
\begin{enumerate}

  \item I would write the GUI---i.e. the window where the user enters
    options---last.  First set the options by hand in the code, or
    using simple text prompts.  Don't code up the GUI until you're
    sure of the options you want to include in the it, since it can be
    a pain to add an option or some text once you've already gotten
    the spacing and such to work.

  \item Depending on how you set thing up you may want to get a random
    number between $0$ and $1$, instead of an integer in a range,
    which is what randrange gives you.  To get a random floating point
    number between $0$ and $1$, use the random() function:
\begin{verbatim}
    from random import random
    x = random()
\end{verbatim}
$x$ is now a random number between $0$ and $1$. 

  \item You may also want the computer to pause between drawing random
    circles.  (I did this for my program, since I didn't like it when
    python drew the circles really quickly.)  Here's how you can get
    the program to pause:
\begin{verbatim}
    from time import sleep
    sleep(1)
\end{verbatim}
The function sleep(1) will make the program go to sleep---i.e. do
nothing---for 1 second.  In general, sleep(n) will make the program
sleep for n seconds.  

\end{enumerate}



\end{document}
