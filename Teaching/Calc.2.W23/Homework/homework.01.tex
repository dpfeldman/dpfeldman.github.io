\documentclass[12pt]{article}
\oddsidemargin=0.0in
\evensidemargin=0.0in
\textwidth=6.5in
\topmargin=-0.35in
\textheight=9in
\usepackage{hyperref}
 
\begin{document}
\pagestyle{empty}
 
\begin{center}
{\LARGE {\bf Homework One}}\\
\bigskip
{\Large {\bf Calculus II}}\\
\bigskip
{\Large {\bf College of the Atlantic}}\\
\bigskip
{ {\bf Due Friday, January 13, 2023}}\\ 
\end{center}
\medskip


\noindent There is only one part to this assignment.\\

\noindent {\bf Part 1: WeBWorK}.  Do Homework 00 and 01 which you
will find the WeBWorK page here:
\url{https://webwork-hosting.runestone.academy/webwork2/coa-feldman-es3012m-winter2023}.

%I recommend doing the WeBWorK part of the homework first.  This will
%enable you to benefit WeBWorK's instant feedback before you do part
%two.\\ 


%\noindent {\bf Part 2: Non-WeBWorK problems}.  Here are some
%instructions for how to submit this part of the assignment.
%\begin{itemize}
%\item Do the problems by hand using pencil (or pen) and paper.
%  There is no need to type this assignment.
%\item If you like working on a tablet, go for it. 
%\item Make a pdf scan of your work using genius scan or some
%  similar scanning app.  Please make the homework into a single
%  pdf, not multiple pdfs.
%\item Submit the assignment on google classroom.  Please don't
%  email it to me.  (Between my two classes I will be receiving
%  around 60 assignments a week.  Keeping track of them all in email 
%  is challenging.)
%\item If you want, you can do the non-WeBWorK problems in pairs and
%  submit only one assignment for the two of you. \\
%\end{itemize}

%\noindent Here are some non-WeBWorK problems.

%\begin{enumerate}
%\setlength{\itemsep}{-1mm}
%  \item Determine an equation for the linear function that generates
%    the values in the table below.  

%\begin{center}
%\begin{tabular}{|| l | l ||}
%\hline $x$ & $f(x)$ \\
%\hline
%5.2 & 27.8 \\
%5.3 & 29.2 \\
%5.4 & 30.6 \\
%5.5 & 32.0 \\
%5.6 & 33.4 \\
%\hline
%\end{tabular}
%\end{center}

%\item The graph of Fahrenheit temperature, F, as a function of Celsius
%  temperature, C, is a line.  We know that 212 F and 100 C are the
%  same; this is the temperature at which water boils (under standard
%  pressure).  And we also know that 32 F and 0 C are the same; this is
%  the temperature at which water freezes.
%  \begin{enumerate}
%  \item What is the slope of the graph?
%  \item What is the equation of the line?  (I.e., F as a function of
%    C.)
%  \item Use the equation to find what Fahrenheit temperature
%    corresponds to 20 C.
%  \item What temperature is the same number of degrees in both C and
%    F?
%  \end{enumerate}
%
%  \item The cost of planting a crop is usually a function of the
%    number of acres sown. The cost of the equipment is a \emph{fixed
%      cost}, because it must be paid regardless of the number of
%    acres planted. The cost of supplies and labor varies with the
%    number of acres planted and is called the \emph{variable cost}.
%    Supposed the fixed costs of \$10,000 and the variable costs are
%    \$200 per acre. Let $C$ represent the total cost of planting $x$
%    acres of a crop.
%    \begin{enumerate}
%    \item Find a formula for $C$ as a function of $x$.
%    \item Graph the function.
%    \item Which feature on the graph (slope or y-intercept) represents
%      the fixed cost?  Which represents the variable cost?
%    \end{enumerate}

%\end{enumerate}




\end{document}
