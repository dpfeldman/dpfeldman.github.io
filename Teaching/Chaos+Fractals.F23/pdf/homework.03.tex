\documentclass[12pt]{article}
\oddsidemargin=0.0in
\evensidemargin=0.0in
\textwidth=6.5in
\topmargin=-0.35in
\textheight=9in
%\usepackage{doublespace}
\usepackage{hyperref}

\hypersetup{
    colorlinks=true,
    linkcolor=magenta,
    filecolor=magenta,      
    urlcolor=blue,
    %pdftitle={Overleaf Example},
    %pdfpagemode=FullScreen,
    }

\begin{document}
\pagestyle{empty}
 
\begin{center}
{\LARGE {\bf Homework Three}}\\
\bigskip
{\Large {\bf Chaos and Fractals}}\\
\bigskip
{\Large {\bf College of the Atlantic}}\\
\bigskip
{ {\bf Due Friday, September 29, 2023}}\\  
\end{center}
\medskip


\noindent There are two parts to this assignment.\\

\noindent {\bf Part 1: WeBWorK}.  Do Homework 03 which you will find
on your
\href{https://webwork-hosting.runestone.academy/webwork2/coa-feldman-es1026i-fall2023}{WeBWorK
  page}.  I recommend doing the WeBWorK part of the 
homework first.  This will enable you to benefit from WeBWorK's
instant, if not necessarily friendly, feedback before you do part two.\\

\noindent {\bf Part 2: Problems from the Textbook}.  Here are some
instructions for how to submit this part of the assignment.
\begin{itemize}
\item Do the problems by hand using pencil (or pen) and paper.
  There is no need to type of this assignment.
\item Make a pdf scan of your work using genius scan or some
  similar scanning app.  Please make the homework into a single
  pdf, not multiple pdfs.
\item Submit the assignment on google classroom.  Please don't
  email it to me.  (Between my two classes I will be receiving
  around 55 assignments a week.  Keeping track of them all in email
  is challenging.)
%\item Do problem 40 in pairs. Work with someone else in the class
%  and hand in only one solution for the two of you.
%\item If you want, you can do all the problems in pairs and hand
%  in only one set of solutions.\\
\end{itemize}

\noindent There are three textbook problems this week:

\begin{itemize}
\setlength{\itemsep}{-1mm}
\item Chapter 9, problem 9.5
\item Chapter 9, problem 9.7
\item Chapter 11, problem 1
\item \emph{Optional:} Chpater 7, problem 9--11. This problems have
  you investigate fixed points and cycles for the logistic equation
  using a modest amount of algebra. 
\end{itemize}






\end{document}
