\documentclass[12pt]{article}
\oddsidemargin=0.0in
\evensidemargin=0.0in
\textwidth=6.5in
\topmargin=-0.55in
\textheight=9in
%\usepackage{doublespace}
\usepackage{hyperref}

\hypersetup{
    colorlinks=true,
    linkcolor=magenta,
    filecolor=magenta,      
    urlcolor=blue,
    %pdftitle={Overleaf Example},
    %pdfpagemode=FullScreen,
    }

\begin{document}
\pagestyle{empty}
 
\begin{center}
{\LARGE {\bf Logistic Explorations!}}\\
\bigskip
{\Large {\bf Chaos and Fractals}}\\
\bigskip
{\Large {\bf College of the Atlantic}}\\
\bigskip
    {{\bf Do these before class on Friday, September 22, 2023}}\\ 
\end{center}
\medskip

\noindent General Instructions

\begin{itemize}
\setlength{\itemsep}{0mm}
\item Do this with others if you like. It might be more fun that
  way.
\item We'll go over this in discussion section. There's nothing to
  hand in, but take some notes (words and/or pictures) as you
  explore.
\item Don't spend more than 15-20 minutes on these exploration (unless
  you want to).
  \item Use this web page for iterating:
\href{https://dpfeldman.github.io/Chaos/time_series.html}{https://dpfeldman.github.io/Chaos/time\_series.html}
\end{itemize}


\noindent You will investigate iterating the logistic equation, $f(x)
= rx(1-x)$, for different values of the initial condition $x_0$.  For
each of the $r$ values listed below:
\begin{itemize}
  \setlength{\itemsep}{0mm}
  \item Determine the long-term behavior of the itinerary. Does it
    approach a fixed point?  Does it enter in to a cycle?  If so, what
    is the period of the cycle?
  \item Try a few initial conditions for each $r$ value.  Your initial
    condition should be between $0$ and $1$.  Don't choose simple
    fractions like $0.5$ or $0.25$.
  \item For each value, make a rough sketch of the time series plot.\\
\end{itemize}


\noindent Here are the $r$ values to try:
\begin{enumerate}
\item $r=0.5$.
\item $r=1.5$.
\item $r=2.9$ 
\item $r=3.3$.
\item $r=3.5$.
\item $r=3.56$.
\item $r=3.835$.
\item $r=4.0$.
\end{enumerate}


\end{document}
