\documentclass[12pt]{article}
\oddsidemargin=0.0in
\evensidemargin=0.0in
\textwidth=6.5in
\topmargin=-0.55in
\textheight=9.4in
%\usepackage{doublespace}
\usepackage{hyperref}

\hypersetup{
    colorlinks=true,
    linkcolor=magenta,
    filecolor=magenta,      
    urlcolor=blue,
    %pdftitle={Overleaf Example},
    %pdfpagemode=FullScreen,
    }

\begin{document}
\pagestyle{empty}
 
\begin{center}
{\LARGE {\bf Homework Seven}}\\
\bigskip
{\Large {\bf Chaos and Fractals}}\\
\bigskip
{\Large {\bf College of the Atlantic}}\\
\bigskip
{ {\bf Due Friday, October 27, 2023}}\\  
\end{center}
\medskip

\noindent There are two parts to this assignment!\\

\noindent {\bf Part 1: WeBWorK}. Do homework 07 on the 
\href{https://webwork-hosting.runestone.academy/webwork2/coa-feldman-es1026i-fall2023}{WeBWorK
  page}.\\
%I recommend doing the WeBWorK part of the  
%homework first.  This will enable you to benefit from WeBWorK's
%instant, if not necessarily friendly, feedback before you do part two.\\

\noindent {\bf Part 2: Problems from the Textbook}.  Here are some
instructions for how to submit this part of the assignment. 
\begin{itemize}
  \setlength{\itemsep}{-1mm}
\item Do the problems by hand using pencil (or pen) and paper.
  There is no need to type of this assignment.
\item Make a pdf scan of your work using genius scan or some
  similar scanning app.  Please make the homework into a single
  pdf, not multiple pdfs.
\item Submit the assignment on google classroom. %Please don't
%  email it to me.  (Between my two classes I will be receiving
%  around 55 assignments a week.  Keeping track of them all in email
%  is challenging.)
\end{itemize}
\smallskip
\noindent Here are some ``textbook'' problems, which aren't actually
from the textbook.

\begin{enumerate}
\item Consider the following
  complex numbers:
  \begin{equation}
    z_1 = -4 -2i \;,\;\;\; z_2 = 3i \;,\;\;\; z_3 = 2 + 0.5i \;,
    \;\;\;   z_4 = -2 +0.5i \;,
  \end{equation}
  \begin{equation}
  z_5 = -2i \;,\;\;\; z_6 = 2 + 0.5i \;, \;\;\;    z_7 = 4 -2i
  \;,\;\;\; z_8 = -4 -2i \;.
  \end{equation}
    \begin{enumerate}
      \item Plot the above numbers\footnote{Yes, I know
        that there are two numbers that are there twice.} on the complex plane. 
      \item Connect the dots.  The pattern should look familiar.
    \end{enumerate}

  \item Consider the function $f(z) = iz$.
    \begin{enumerate}
      \setlength{\itemsep}{0mm}
    \item Determine first four iterates of $z_0 = 3$.
    \item Determine first four iterates of $z_0 = 2i$.
    \item Plot the iterates for each of the seeds in the complex
      plane.
    \item How would you describe the behavior of the orbits?\\
    \end{enumerate}
      
\end{enumerate}


\noindent Here are textbook problems from the actual textbook:\\

\noindent {\bf Chapter 21:}
\begin{itemize}
\setlength{\itemsep}{-1mm}
\item Chapter 21, problem 1
\item Chapter 21, problem 2
\item Chapter 21, problem 3
\item Chapter 21, problem 4
\item Chapter 21, problem 5 
\end{itemize}

\noindent The textbook problems should be very quick. If they're not
quick, you might be over-thinking things. Check in with one of us for
help! 
\end{document}
