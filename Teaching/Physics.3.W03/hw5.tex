
\documentstyle[12pt]{article}
\oddsidemargin=0in
\textwidth=6.75in
%\topmargin=-.75in
%\textheight=9in
%\usepackage{doublespace}

\renewcommand{\arraystretch}{1.3}

\begin{document}
\pagestyle{empty}

\begin{center}
{\Large {\bf Homework 5}}
\end{center}
\hspace{1cm}\\

\begin{enumerate}

\item Problem 1.1 from McIntyre.

\item Problem 1.3 from McIntyre.

\item Problem 1.5 from McIntyre.


\item Consider the following ket:
\begin{equation}
  | \psi \rangle \, = \, a (\, 2i| + \rangle - i | - \rangle\,) \;.
\label{psi}
\end{equation}
Solve for the $a$ that normalizes $| \psi \rangle$. 

\item Using the normalized $| \psi \rangle$ from the previous problem,
answer the following questions:
\begin{enumerate}
   \item  If the atom is in state $| \psi \rangle$, what is the
    probability of measuring $S_z = 1$?
    \item If the atom is in state $| \psi \rangle$, what is the
    probability of measuring $S_z = -1$?
    \item Do the probabilities add up to $1$?
\end{enumerate}

\item Suppose an atom is in the $|-\rangle_x$ state.  What is the
probability of obtaining a $+$ if $S_z$ is measured?  (Show how this
number arises from the bras and kets, don't just cite the experimental
result.)

\item {\bf (Optional)}
Using the normalized $| \psi \rangle$ from the previous problem,
answer the following questions:
\begin{enumerate}
   \item  If the atom is in state $| \psi \rangle$, what is the
    probability of measuring $S_x = 1$?
    \item If the atom is in state $| \psi \rangle$, what is the
    probability of measuring $S_x = -1$?
    \item Do the probabilities add up to $1$?
\end{enumerate}

\item  {\bf (Optional)} 
\begin{enumerate}
  \item Use complex exponentials to derive formulae for $\cos 3x$ and
  $\sin 3x$.
  \item Use complex exponentials and the binomial theorm to derive
  general expressions for $\cos nx$ and $\sin nx$, where $n$ is a
  positive integer.
\end{enumerate}

\item  {\bf (Optional)} (from p.~369 of Arfkin and Weber, Mathematical
Methods for Physicists, fourth edition, Academic Press, 1995.)  Prove
that:
\begin{enumerate}
\item 
\begin{equation}
\sum_{n=0}^{N-1} \cos nx \, = \, \frac{ \sin N(x/2)}{\sin x/2}
\cos(N-1)\frac{x}{2}\;, 
\end{equation}
and
\item
\begin{equation}
\sum_{n=1}^{N-1} \sin nx \, = \, \frac{\sin N(x/2)}{\sin x/2} \sin(N-1)
\frac{x}{2}\;. 
\end{equation}
\noindent Apparently these sum are used in the analysis of
multiple-slit diffraction patterns. 
\end{enumerate}

\item  {\bf (Optional)} (from p.~369 of Arfkin and Weber, Mathematical
Methods for Physicists, fourth edition, Academic Press, 1995.)
For $-1 < p < 1$, prove that
\begin{enumerate}
\item
\begin{equation}
  \sum_{n=0}^{\infty} \, = \, p^n \cos nx \, = \, \frac{ 1 - p \cos
  x}{ 1 - 2p \cos x + p^2} \;, 
\end{equation}
and,
\item
\begin{equation}
  \sum_{n=0}^\infty p^n \sin nx \, = \, \frac{ p \sin x}{1 - 2p\cos x
  + p^2} \;. 
\end{equation}
These sums apparently are used in the theory of the Fabry-Perot
interferometer.   
\end{enumerate}

\end{enumerate}

\noindent {\bf Hints for the last two:}
\begin{enumerate}

\item Stay calm.  The expressions are large, but I don't think they
require any math that's too fantastic or unthinkable.

\item For each, combine the two sums using complex exponentials:
$e^{ix} = \cos x + i \sin x$. 

\item You'll then need to use the following result about geometric
series:
\begin{equation}
  \sum_{n=1}^\infty a r^{n-1} \, = \, \frac{a}{1-r} \;,
\end{equation} 
for $-1 < r < 1$. 


\end{enumerate}


\end{document}


