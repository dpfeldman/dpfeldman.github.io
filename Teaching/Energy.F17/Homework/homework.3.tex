\documentclass[11pt]{article}
\oddsidemargin=0.0in
\evensidemargin=0.0in
\textwidth=6.5in
\topmargin=-.45in
\textheight=9in
\usepackage{url}
 
\begin{document}
\pagestyle{empty}
 
\begin{center}
{\Large {\bf Homework Three}}\\
\smallskip
{\large {\bf Physics \& Mathematics of Sustainable Energy}}\\
\smallskip
{\large {\bf College of the Atlantic}}\\
\smallskip
{ {\bf Due Tuesday, September 29, 2017}}\\
\end{center}

\medskip
\noindent Please print out this cover sheet and attach it to your
problem solutions.  Completed assignments should go in my mailbox or
be handed in during class.  Please don't hand them to me other times,
as I might end up losing them and that would make us both sad. \\
\medskip

\noindent {\bf Your Name:} \underline{\hspace{5cm}}\\

\noindent {\bf Please list all the students you collaborated with on
  this assignment:}\\[10pt]
\underline{\hspace{7.1cm}} \hspace{2cm}
\underline{\hspace{7.1cm}}\\[10pt]
\underline{\hspace{7.1cm}} \hspace{2cm}
\underline{\hspace{7.1cm}}\\[10pt]
\underline{\hspace{7.1cm}} \hspace{2cm}
\underline{\hspace{7.1cm}}\\[10pt] 
%\bigskip

\noindent {\bf Did you get help from Aura or Morgan? }\\

\noindent {\bf Did you consult any resources other than our textbook
  or class notes?  (If yes, please include citations in your
  solutions.)} \\

\noindent {\bf Were you able to get enough help so you could complete
  this assignment to your satisfaction?}\\

\noindent {\bf Approximately how many hours did you spend on this
  assignment?}\\ 

\noindent {\bf Anything else of note about this assignment?  (It was
  too hard, too easy, lots of fun, too repetitious...)}\\

\bigskip
\bigskip
\bigskip
\bigskip
\noindent The work I am turning in for this assignment is an accurate
reflection of my own understanding of the material.\\[14pt]

\noindent Signature: \underline{\hspace{7cm}} \hspace{1cm} Date:
\underline{\hspace{5cm}} 

{\bf Assignment is on the next page....}

\hspace{2mm}\\
\newpage
\begin{center}
{\Large {\bf Homework Two}}\\
\smallskip
{\large {\bf Physics \& Mathematics of Sustainable Energy}}\\
%\smallskip
%{\large {\bf College of the Atlantic}}\\
\smallskip
{ {\bf Due Tuesday, April 12, 2016}}\\
\end{center}

\begin{enumerate}



\item There is currently an $800$ acre fire burning in
  Tennessee\footnote{\url{www.local8now.com/content/news/Forest-fire-burning-near-Newport-in-Cherokee-National-Forest-374540001.html}}.
  Convert this area to something meaningful---something that can be
  visualized in some way.  Probably this means figuring out the side
  of a square that has the same area as $800$ acres\footnote{Use
    meters, miles, yard, kilometers: whatever seems appropriate and is
    the most meaningful to you}.  But you could also compare it to the
  area of MDI or your hometown or a football field or whatever.

\item How much area would it take to generate all of your energy needs
  from electricity generated on a terrestrial wind
  farm?\footnote{Assume that you are an American. Or that you are
    consuming at the rate of a typical American while living in the
    US.} 

\item Optional: how much land is required to grow the food that you
  eat?  I have no idea what the answer to this is, but I'd be curious
  to find out.  Probably different people have estimated this in
  different ways, and surely it depends a lot on ones diet.  I'd be
  interested to see what estimates are out there.  If you find any
  (semi)reliable references for this, please let me know.  (Again,
  this problem is 100\% optional.)

\item Suppose that a certain wind turbine generates a certain amount
  of energy per month.  What would happen to the energy generated per
  month if:
\begin{enumerate}
\setlength{\itemsep}{-1mm}
  \item The diameter of the blades was increased by 20\%?
  \item The turbine was re-located someplace where the average
    wind speed was 30\% higher?
  \item The turbine was re-located someplace where the density of the
    substance flowing around it was twice as large?
\end{enumerate}

\item In this problem we'll think some about off-shore wind in the
  Gulf of Maine.
\begin{enumerate}
  \item First, let's collect some facts:  Write down the following
    values.  (All of these figures should be in your class notes.
    They're also in MacKay and also my book.)
\begin{enumerate}
    \item The average total energy consumption per person in the US,
      in units of kWh per person per day.
    \item The worldwide average emissions per person per year, in tons
      of CO$_2$ equivalent. 
    \item The average emissions per person per year of the average
      American, in tons of CO$_2$ equivalent.
    \item The average amount of CO$_2$ released per kWh of electricity 
      generated. 
\end{enumerate}

\item What power would be needed to provide the total energy needs of
  everyone in Maine? 

\item Would would be the area needed for an offshore windfarm that
  could deliver this power.  Assume that the offshore windfarm
  generates $3$ Watts per square meter.  Express your answer in km$^2$
  and mi$^2$.

\item What is side of a square whose area is equal to the area you
  found in the above problem?

\item Find and print out map of New England that includes the Gulf of
  Maine.  Be sure this map includes a scale.  Draw on this map a
  square that is the size of the square you calculated in the previous
  question.  Be reasonably careful when you draw the square, but don't
  stress out about getting the size super accurate.  

\item Assuming that this electricity from wind replaced ``average''
  US electricity, how much CO$_2$ has been prevented from being
  released into the atmosphere.  Express your answer in terms of tons
  of CO$_2$ per Mainer per year. Is this a lot or a little? 

\end{enumerate}


\item Optional, but recommended: In class we derived the following:
\begin{equation}
 {\rm wind\;farm\;power\;density} \, = \frac{\pi}{200}\frac{1}{2} \rho
 v^3 \;,
\end{equation}
where $v$ is the windspeed and $\rho$ is the density of air.  If the
average windspeed is $6$ m/s, what is the power density of the wind
farm?  (The density of air is $\rho = 1.225$ kg/m$^3$.)

\end{enumerate}


\end{document}
