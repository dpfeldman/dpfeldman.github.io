\documentclass[12pt]{article}
\oddsidemargin=0in
\textwidth=6.25in
%\topmargin=-.75in
%\textheight=9in
\usepackage{epsf}

\renewcommand{\arraystretch}{1.3}

\begin{document}
\pagestyle{empty}

\begin{center}
{\large {\bf Lab 5: Rolling Stuff and Specific Heat}}\\
\smallskip 
\end{center}

\hspace{2mm}\\

\noindent {\bf Rolling Stuff}\\ 

Consider a disk and a hoop. Suppose the objects are released from rest
and allowed to roll down an incline. The incline has length L and one
end is raised to a height of $h$. 
\begin{enumerate}
\item Determine an expression for the velocity of the hoop at the
  bottom of the incline. Your expression will depend on $h$. 
\item Determine a similar expression for the velocity of the disk.
\item Calculate the ratio of the two velocities. 
\item Make reasonably careful measurements of the velocities of the
  hoop and the disk at the bottom of the incline. (Take several
  measurements and average your results.)  
\item Does your experimental velocity ratio agree with the theoretical
  velocity ratio? \\
\end{enumerate}


\noindent {\bf Specific Heat}\\

Suppose you place a hot piece of metal of mass $m$ in a Styrofoam cup
containing some water of mass $M$.  Let the initial temperature of the
metal be $T_m$ and the initial temperature of the water be $T_w$.  By
how much will the temperature of the water increase?  Try this out
with one of the pieces of metal.  Does your measured temperature
increase agree with what you calculated?



\end{document}
