\documentclass[12pt]{article}
\oddsidemargin=0in
\textwidth=6.25in
%\topmargin=-.75in
%\textheight=9in
\usepackage{epsf}

\renewcommand{\arraystretch}{1.3}

\begin{document}
\pagestyle{empty}

\begin{center}
{\large {\bf Lab 4: Conservation of Energy, Estimation}}\\
\smallskip 
{\large {\bf and the Beginning of Rotation}}\\
\end{center}

\hspace{2mm}\\

\noindent {\bf More Estimation Practice}\\

\noindent Estimate the following quantities.  Aim to get results
accurate to within an order of magnitude (i.e., within roughly a power
of ten.)
\begin{enumerate}
  \setlength{\itemsep}{-1mm}
     \item How many people visit our campus from cruise ships every
     fall?   
     \item When you fly from New York to Los Angeles you go through
     three time zones.  The distance of the journey is around $3000$
     miles.  Use this information to estimate the radius of the
     earth.  \\
\end{enumerate}

\noindent {\bf Cartoon Physics Exercises}\\

\noindent There are several short physics cartoon questions or
puzzles. Work through the questions, and then see me and check your
answers. \\



\noindent {\bf Pieces of Pie}\\

\noindent Ask me or Cecily for the worksheet and we'll give you
directions.\\ 

\noindent {\bf Rolling Stuff}\\

\noindent  Consider a hoop and a disk of equal mass and equal radius.
\begin{enumerate}
\setlength{\itemsep}{-1mm}
  \item If you roll both down an incline, which will get to the bottom
  first, or will they get there at the same time?

  \item Do the experiment.  Tilt a table and roll the objects.  Please
  be careful to catch the objects before they roll off the table.

  \item What happened?  How do you explain this?
\end{enumerate}

\end{document}
