\documentclass[12pt]{article}
\oddsidemargin=0in
\textwidth=6.25in
%\topmargin=-.75in
%\textheight=9in
\usepackage{epsf}

\renewcommand{\arraystretch}{1.3}

\begin{document}
\pagestyle{empty}

\begin{center}
{\large {\bf Lab 1:  Vectors}}\\
\end{center}

\noindent {\bf General Instructions:}
General instructions.
\begin{enumerate}
\setlength{\itemsep}{0mm}
  \item Work in groups of two or three.
  \item Please check with me or one of the TAs before going on to the
    next exercise.
  \item Please hand in only one write-up per group.\\
\end{enumerate}



\noindent Consider again the following vectors:
\begin{itemize}
\setlength{\itemsep}{0mm}
  \item $\vec a \, = \, $ the displacement from COA to the Bar Harbor
  airport.

  \item $\vec b \, = \, $ the displacement from MDI High School to
  Thunder Hole.

  \item $\vec c \, = \, $ the displacement from Somesville to The
  Jackson Lab.

\end{itemize}

\begin{enumerate}
\item Specify vectors $\vec a, \vec b, \vec c$ by giving their
components.  Do not use trigonometry.

\item Use a ruler and a protractor to draw (to scale) the following:
\begin{enumerate}
\setlength{\itemsep}{0mm}
  \item $ \vec{v} = 3\vec{a}$
  \item $ \vec{u} = \vec{b} + \vec{c}$
\end{enumerate}

\item By measuring, determine the components of:
\begin{enumerate}
\setlength{\itemsep}{0mm}
  \item $\vec{a}$
  \item $\vec{b}$
  \item $\vec{c}$
  \item $\vec{u}$
  \item $\vec{v}$
\end{enumerate}

\item What is the procedure for adding two vectors using component
  form

\item What is the procedure for scalar multiplication of vectors using
  component form? 

\item Check with me or a TA.  We'll ask you some questions and then
  give you the next sheet.  

\end{enumerate}




%<h3>A little brain teaser</h3>
%You have a drawer containing 8 blue socks, 10 green socks, and 12
%orange socks.  You would like to wear a matching pair of socks.  You
%grab socks one at a time from the draw.  What is the maximum number of
%socks you need to draw before you have a matching pair in your hands?<p>


\newpage 


\begin{center}
  {\bf Right Triangles and Ratios}\\
\end{center}

\hspace{2mm}\\

\begin{enumerate}
\setlength{\itemsep}{0mm}
  \item Using a ruler measure the $p$'s and $q$'s on each of the
    triangles in Fig.~\ref{triangle.1}.
  \item Determine values of $p_1/q_1$, $p_2/q_2$, and $p_3/q_3$.
  \item Then do the same for the triangle on Fig.~\ref{triangle.2}.
  \item What does this ratio $p/q$ tell you?
  \item Invent a name for this ratio.  
  \item Get sheet three from Sanjeeva or me and do the problems on it. 

\end{enumerate}

% **********************************************************************
\begin{figure}[hbp]
%\vspace{-0.2in}
%\hspace{0.2in}
\epsfxsize=5.8in
\begin{center}
\leavevmode
\epsffile{triangle.1.eps}
\end{center}
\vspace{-2mm}
\caption{}
\label{triangle.1}
\end{figure}
% **********************************************************************
\newpage
% **********************************************************************
\begin{figure}[ht]
%\vspace{-0.2in}
%\hspace{0.2in}
\epsfxsize=5.8in
\begin{center}
\leavevmode
\epsffile{triangle.2.eps}
\end{center}
\vspace{-2mm}
\caption{}
\label{triangle.2}
\end{figure}
% **********************************************************************




\newpage

\begin{center}
  {\bf Trigonmetry}\\
\end{center}

\hspace{2mm}\\

\begin{enumerate}

\item Consider a vector $\vec{a}$ which is a $10$ meter displacement,
  $37$ degrees north of west.  And let $\vec{b}$ be a $20$ meter
  displacement $45$ degrees west of south.

\item Write $\vec{a}$ and $\vec{b}$ in component form.  Use
  trigonmetry. 

\item Determine the following:
\begin{enumerate}
\setlength{\itemsep}{0mm}
  \item $\vec{a} - 2\vec{b}$
  \item $3\vec{a}$
  \item $5 \vec{a} + 3 \vec{b}$
\end{enumerate}
express your answers both in component form and magnitude-direction
form. 


\end{enumerate}

\newpage

\begin{center}
  {\bf Sextants }\\
\end{center}

\hspace{2mm}\\

\begin{enumerate}

\item Trigonometry Warm Up:
\begin{enumerate}
\setlength{\itemsep}{0mm}
  \item You stand 50 meters away from a flag pole.  You have to look
  at an angle of 53 degrees from the horizon to see the top of the
  pole.  What is the pole's height? 

  \item You stand 75 meters away from a tree that's 100 meters tall.
  At what angle must you tilt your head so that you look straight at
  the top of the tree? 

\end{enumerate} 

\item Trigonometry and Trees:

\begin{enumerate}
\setlength{\itemsep}{0mm}
  \item Grab a sextant.  Go outside and figure out how to use it.
  (Talk to or Cecily.) 

  \item Measure the height of the large pine tree on the North end of
  the field between the Blair/Tyson and the arts and sciences
  building.  
\end{enumerate}

\end{enumerate}


\end{document}

