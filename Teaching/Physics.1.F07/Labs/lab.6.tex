\documentclass[12pt]{article}
\oddsidemargin=0in
\textwidth=6.25in
%\topmargin=-.75in
%\textheight=9in
\usepackage{epsf}

\renewcommand{\arraystretch}{1.3}

\begin{document}
\pagestyle{empty}

\begin{center}
{\large {\bf Lab 6: Angular Momentum and Power}}\\
\smallskip 
\end{center}

\hspace{2mm}\\

\noindent {\bf Angular Momentum}\\
 
\noindent {\bf Be careful!} It's possible to lose your balance and
fall down while doing these. Do the following experiments.  First
predict, using the idea of conservation of angular momentum, what will
happen. (Each person in your group should do at least one of these. If
no other groups are waiting for the rotating platform, everyone should
try each.)  
\begin{enumerate}
   \item Stand on the ground. Spin of the bicycle wheel and hold it
   horizontally. Then get on the platform. Turn the bicycle wheel
   upside down. 
   \item Stand on the platform and hold the wheel. (The wheel
   shouldn't be spinning yet.) Then, hold the wheel horizontally and
   give it a spin. 
   \item Hold the two heavy weights and stand on the platform. Have
   someone give you a gentle spin. Move the weights in and out. 
   \item Stand on the platform and hold something heavy and
   unbreakable. 
   \begin{enumerate}
   \item Throw the heavy thing in such a manner that you end up
     rotating after the throw. 
   \item Throw the heavy thing equally hard, but now throw it
     so that you don't rotate after the throw. \\ 
   \end{enumerate}
\end{enumerate}



\noindent {\bf Cartoon Physics}\\

\noindent Consider the riddles posed on the handout.  Ponder.
Discuss.  Come up with an answer.  Then check your answer with me or
on the answer sheet.\\


\newpage
\noindent {\bf Efficiency}\\

\noindent A watt-meter is a device that measures how much power is
being drawn by an electrical appliance.  Take a look at the watt-meter
and see how it works.  We will use the watt-meter to measure the
efficiency of various water-heating devices.
\begin{enumerate}
  \item Measure some water into a styrofoam cup and place the cup in a
  microwave.  Determine the temperature of the water.  Turn the
  microwave on for a minute or so and measure temperature of the water
  and the amount of energy that the microwave used.  How efficient is
  the microwave.

  \item Measure some water and put it into the electric kettle.
  Eventually, the water will come to a boil.  Once the water starts
  boiling, keep track of the energy used by the kettle.  Let the water
  boil for five minutes or so.  Then re-measure the mass of the
  water. (It should be less, since some has boiled away.)  How
  efficient is the kettle?
\end{enumerate}



\end{document}
