\documentclass[12pt]{article}
\oddsidemargin=0in
\textwidth=6.25in
%\topmargin=-.75in
%\textheight=9in
\usepackage{epsf}

\renewcommand{\arraystretch}{1.3}

\begin{document}
\pagestyle{empty}

\begin{center}
{\large {\bf Lab 2: Momentum and more Vectors}}\\
\end{center}

\hspace{2mm}\\

\noindent {\bf Air Tracks}

Please be careful with the airtracks.  Although it doesn't look like
it, they're actually somewhat fragile.  If the track or the metal
carts get bent, they glide well and the experiments won't work.  Also,
please do not rest the carts on the air track when it's not on.  Turn
the vacuum on and then put the carts on.  And take the cars off before
turning it off.  Thanks.

Try out the following things.  Make qualitative observations.  Don't
aim for quantitative, precise results. 

\begin{enumerate}
  \setlength{\itemsep}{-1mm}
    \item Level the air track.
    \item Take a cart and give it a shove. What happens to the cart?
    What interactions is the cart participating in? 
    \item Reproduce the experiment discussed in Figure C3.3.
    \item Reproduce the experiment discussed in Figure C3.4.
    \item Reproduce the experiment discussed in Figure C3.5.
    \item Imagine a situation where a cart collides with another cart
    of equal mass that's at rest, and they stick together. What do you
    think will happen? Be sure to write down your prediction. Then do the
    experiment. Use duct tape to make the carts stick.  
    \item Repeat the above, but have a small car hit a light cart at
    rest.
   \item Repeat again, but have a large cart his a small cart at
   rest. \\
\end{enumerate}


\noindent {\bf A  Brain Teaser}

You have a drawer containing 8 blue socks, 10 green socks, and 12
orange socks.  You would like to wear a matching pair of socks.  You
grab socks one at a time from the draw.  What is the maximum number of
socks you need to draw before you have a matching pair in your hands?
(This has nothing to do with physics.) 



\newpage

\begin{center}
  {\bf Sextants }\\
\end{center}

\hspace{2mm}\\

\begin{enumerate}

\item Trigonometry Warm Up:
\begin{enumerate}
\setlength{\itemsep}{0mm}
  \item You stand 50 meters away from a flag pole.  You have to look
  at an angle of 53 degrees from the horizon to see the top of the
  pole.  What is the pole's height? 

  \item You stand 75 meters away from a tree that's 100 meters tall.
  At what angle must you tilt your head so that you look straight at
  the top of the tree? 

\end{enumerate} 

\item Trigonometry and Trees:

\begin{enumerate}
\setlength{\itemsep}{0mm}
  \item Grab a sextant.  Go outside and figure out how to use it.
  (Talk to or Cecily.) 

  \item Measure the height of the large pine tree on the North end of
  the field between the Blair/Tyson and the arts and sciences
  building.  
\end{enumerate}

\end{enumerate}

\newpage

{\bf Scientific Notation and Estimation}\\

\noindent First, some practice with scientific notation.
\begin{enumerate}
  \setlength{\itemsep}{-1mm}
  \item Convert the following numbers into scientific notation.
  \begin{enumerate}
    \setlength{\itemsep}{-1mm}
    \item $123456$
    \item $0.000045$
    \item 77 billion
    \item $945.670$
\end{enumerate}
  \item Convert the following from scientific notation into regular
  decimals. 
  \begin{enumerate}
    \setlength{\itemsep}{-1mm}
    \item $3.4 \times 10^3$
    \item $3.4 \times 10^{-3}$
    \item $6 \times 10^6$\\
  \end{enumerate}

\end{enumerate}


\noindent Next, some estimation, or ``Fermi'' problems.  I'm looking
for order of magnitude estimates, not precise answers! (This means
correct to within a factor of ten.)  You should round off with
reckless abandon, and avoid using a calculator if at all possible.  
\begin{enumerate}
  \setlength{\itemsep}{-1mm}
     \item Estimate the speed, in mph, at which human hair grows.
     \item Estimate the average speed, in mph, at which a human grows
   from age 0 to 18. 
      \item Estimate the number of pounds of meat consumed at our
      dining hall every week. 
\end{enumerate}
\end{document}

