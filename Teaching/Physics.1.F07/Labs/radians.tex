
\documentclass[12pt]{article}
\oddsidemargin=0in
\textwidth=6.75in
%\topmargin=-.75in
\textheight=9in
\usepackage{epsf}

\renewcommand{\arraystretch}{1.3}

\begin{document}
\pagestyle{empty}


\begin{center}
{\bf {\Large Slices of Pie}}\\
\end{center}

\begin{enumerate}
\setlength{\itemsep}{0mm}
\item For each slice and sub-slice, measure the length of the edge of
the slice and the length of the crust.

\item Calculate the ratio of the crust length to the side length
(i.e. crust length divided by side length?) for each slice.

\item What is the physical meaning of this number?  

\item See Cecily or me and we'll ask you some questions and then give
  you the next sheet.

\end{enumerate}

 
% **********************************************************************
\begin{figure}[hbp]
%\vspace{-0.2in}
%\hspace{0.2in}
\epsfxsize=6.5in
\begin{center}
\leavevmode
\epsffile{circles.eps}
\end{center}
\vspace{2mm}
%\caption{An unknown function.}
%\label{line}
\end{figure}
% **********************************************************************

\newpage


\begin{center}
{\bf {\Large Slices of Pi}}\\
\end{center}

\hspace{2mm}\\

\begin{enumerate}

\item You now have discovered a new way of measuring angles.  The
  usual name for this new angle measure is {\em radians}.  But you can
  call it something else if you want.  Come up with a formula that
  relates radians to degrees.

\item How many radians are in ninety degrees?

\item How many degrees are there in 4 radians?


\end{enumerate}

\end{document}


