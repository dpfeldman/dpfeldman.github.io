\documentclass[12pt]{article}
\oddsidemargin=0in
\textwidth=6.25in
%\topmargin=-.75in
%\textheight=9in
\usepackage{epsf}

\renewcommand{\arraystretch}{1.3}

\begin{document}
\pagestyle{empty}

\begin{center}
{\large {\bf Lab 7: Motion}}\\
\smallskip 
\end{center}

\hspace{2mm}\\

\noindent {\bf The Dot Machine}

\begin{enumerate}
   \item Make a motion diagram for something falling using the ticker
   timer. 
   \item Draw the velocity and acceleration arrows. Do this for around
   10 dots.  Choose a region of dots for which the acceleration appears
   close to constant.
   \item Then make a plot of the position, velocity, and acceleration
   vs. time. Use a different set of axes for each plot. \\ 
\end{enumerate}

\noindent {\bf Motion Worksheet}\\

\noindent Do the worksheet on graphical descriptions of motion.\\

\hspace{2mm}\\

\noindent {\bf Measuring g}\\

\noindent We will try to calculate $g$, the acceleration due to
gravity, by measuring the time it take for an object to fall.
To do so, time how long it takes for a quarter to fall exactly $1.5$
meters.  Practice using the stopwatch so you get a feel for how long
it takes to start and stop it.  See me or Cecily for details.

We will see in class tomorrow that for an object with a constant
acceleration of $a$, it's position $y$ will be given by
\begin{equation}
  y \, = \, \frac{1}{2}gt^2 \;.
\end{equation}
I.e., the above formula tell us how far the object will fall, if it
accelerates for $t$ seconds at a constant rate of $a$. 

\newpage

Name:\\


\begin{center}
\begin{tabular}{|| c | c ||}
\hline Trial & Falling time $t$ \\
\hline
%-6 &  1 \\
1 & \\
2 & \\
3 & \\
4 & \\
5 & \\
\hline
\end{tabular}
\end{center}

\hspace{2mm}\\

Average your results to come up with your best estimate for the
acceleration. 

\hspace{2mm}\\
\hspace{2mm}\\
\hspace{2mm}\\
\hspace{2mm}\\

Name:\\


\begin{center}
\begin{tabular}{|| c | c ||}
\hline Trial & Falling time $t$ \\
\hline
%-6 &  1 \\
1 & \\
2 & \\
3 & \\
4 & \\
5 & \\
\hline
\end{tabular}
\end{center}

\hspace{2mm}\\

Average your results to come up with your best estimate for the
acceleration. 



\end{document}
