\documentclass[12pt]{article}
\oddsidemargin=0.0in
\evensidemargin=0.0in
\textwidth=6.5in
%\topmargin=-.75in
%\textheight=9in
%\usepackage{doublespace}
 
\begin{document}
\pagestyle{empty}
 
\begin{center}
{\large {\bf Physics I}}\\
\medskip
{\large {\bf Homework Six}}\\
\medskip
{ {\bf Due Friday 28  October, 2011}}\\
\end{center}

\hspace{2mm}\\

\noindent {\bf Chapter C11:}

\begin{enumerate}
\setlength{\itemsep}{-1mm}
  \item C11B.4--8
  \item C11S.5
  \item C11R.1\\
%  \item Estimate how long it would take to melt 0.25 kg of ice in a
%    toaster oven.  (Note:  I do not recommend trying this.  If you
%    can't resist, please put the ice in something so that the water
%    doesn't drip everywhere and short out the toaster, causing a fire
%    or explosion or something.)
\end{enumerate}


\noindent {\bf Chapter C12:}

\begin{enumerate}
\setlength{\itemsep}{-1mm}
  \item A refrigerator might draw 350W of power when it is on.  Assume
    that it is on for a quarter of the time.  
  \begin{enumerate}
  \setlength{\itemsep}{-1mm}
    \item How much energy, in units of kWh, would this refrigerator
      use in one month? 
    \item How much would this cost in Maine?
  \end{enumerate}
  \item C12B.5
  \item C12S.3  \\

  %\item  What wattage light bulb uses 1 kWh in one day?

  \item In a typical day a typical person typically eats around 2500
    calories of food.   These are dietary
    calories. Confusingly, 1 dietary calorie equals 1000 ``real''
    calories.  One ``real'' calorie is equal to $4.18$ Joules. 
\begin{enumerate}
    \item  How many Joules does a typical person consume in a day?
    \item What power is this? Express your answer in kW.
    \item Most of the food energy you consume ultimately gets
      converted to heat. Thus, we can view people as heaters---they
      convert chemical food energy into thermal energy. How many
      people would you need to have in a room to have a heating power
      roughly equivalent to one 1500 W space heater?
\end{enumerate}

\end{enumerate}

\noindent {\bf Chapter C13:}

\begin{enumerate}
\setlength{\itemsep}{-1mm}
  \item C13B.2
  \item C13B.5
  \item C13B.7
  \item C13B.8
\end{enumerate}

%\noindent {\bf Chapter C14:}
%
%\begin{enumerate}
%\setlength{\itemsep}{-1mm}
%  \item C14S.2
%  \item C14S.3
%  \item C14S.4
%\end{enumerate}



\end{document}
