\documentclass[12pt]{article}
\oddsidemargin=0.0in
\evensidemargin=0.0in
\textwidth=6.5in
\topmargin=-.35in
\textheight=8.8in
%\usepackage{doublespace}
 
\begin{document}
\pagestyle{empty}
 
\begin{center}
{\large {\bf Calculus II}}\\
\medskip
{\large {\bf Spreadsheets, Definite Integrals, and Left- and
    Right-hand Sums}}\\ 
\medskip
{ {\bf Due Monday January 16, 2011}}\\ 
\end{center}

\hspace{2mm}\\

\noindent {\bf Purpose}
\begin{enumerate}
  \item To gain a deeper understanding of left- and right-hand sums
    and to learn how to approximate a definite integral using a
    spreadsheet.  (It is not unusual to have to use a computer to
    evaluate an integral, so this is more than just an idle exercise.)
  \item To gain experience working with spreadsheets, especially
    entering formulas.  Using spreadsheets is an incredibly valuable
    skill in science, math, and in life in general.\\
\end{enumerate}


\noindent {\bf General Instructions}

\begin{itemize}
\setlength{\itemsep}{-1mm}
  \item You can do this exercise in pairs if you want.  If you do, you
    should be sure to take turns typing, so you each gain direct
    experience working with the spreadsheet.  
  \item Only hand in one assignment per duo.
  \item This is not meant to be an ordeal.  If it takes much longer
    than an hour, stop.  We can go over it when I get back if needed. 
  \item I assume you have access to a spreadsheet program of some
    sort.  Openoffice Calc is a good free option, as is google
    spreadsheets.  (To find google spreadsheets, go to google docs and
    choose spreadsheet.  You'll need a google account.)\\
\end{itemize}




\noindent {\bf Formulas and Spreadsheets}
\begin{itemize}
  \item In addition to entering numbers into spreadsheets, it is
    possible to enter formulas.  This is incredibly useful and handy.
  \item It is not hard.  To indicate that what you're typing into the
    cell is a formula, begin with an ``=''.
  \item Refer to the values in other cells by typing \$ followed by
    the cell name.
  \item For example, a formula that takes the value in cell A2 and
    adds it to the value of the cell in B3 would be: {\tt =\$A2+\$B3}.
  \item If this isn't clear, ask someone who has experience with
    formulas and spreadsheets to show you.
\end{itemize}

\noindent {\bf Initial Example}\\

\noindent We will approximate the definite integral we looked at in
class on Friday: 
\begin{equation}
  \int_1^3 x^2 \, dx \; .
\end{equation}
Let's use $\Delta t = 0.5$.  The LH sum is:
\begin{equation}
  {\rm LH\;sum} \, = \, (1)^2\Delta t + (1.5)^2\Delta t + 
    (2.0)^2\Delta t + (2.5)^2\Delta t \;.
\end{equation}
Evaluating this one gets $6.75$.  Similarly, the RH sum is:
\begin{equation}
  {\rm RH\;sum} \, = \, (1.5)^2\Delta t + (2.0)^2\Delta t + 
    (2.5)^2\Delta t + (3.0)^2\Delta t \;.
\end{equation}
The RH sum evaluates to $10.75$.

I have prepared a sample spreadsheet that does this calculation for
you.  Take a look at the spreadsheet and see how it works.  (There are
many ways to accomplish this task.  If another way seems easier to
you, go for it.)  You can download the 
sample spreadsheet from the homework page on the course website.  You
can upload the excel document to google docs if you want.\\

\noindent {\bf Stuff to do:}

\begin{enumerate}
\setlength{\itemsep}{-1mm}
  \item Modify the spreadsheet I sent you to compute left and right
    hand sums for $\Delta t = 0.25$.
  \item Modify the spreadsheet I sent you to compute left and right
    hand sums for $\Delta t = 0.1$.
  \item Use a spreadsheet to estimate:
\begin{equation}
  \int_2^{10}e^{-x} \, dx \;.
\end{equation}
  \item Use a spreadsheet to estimate:
\begin{equation}
  \int_{-2}^2\sin(x) \, dx \;.
\end{equation}

\end{enumerate}



 
\end{document}
