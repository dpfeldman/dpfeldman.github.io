\documentclass[12pt]{article}
\oddsidemargin=0.0in
\evensidemargin=0.0in
\textwidth=6.5in
\topmargin=-0.55in
\textheight=9.3in
\usepackage{hyperref}
 
\begin{document}
\pagestyle{empty}
 
\begin{center}
{\LARGE {\bf Homework Three}}\\
\bigskip
{\Large {\bf Calculus I}}\\
\bigskip
{\Large {\bf College of the Atlantic}}\\
\bigskip
{ {\bf Due Friday, September 30, 2022}}\\ 
\end{center}
\medskip


\noindent There are two parts to this assignment.\\

\noindent {\bf Part 1: WeBWorK}.  Do Homework 03A and 03B on WeBWorK.  The WeBWorK page is here:
\url{https://webwork.runestone.academy/webwork2/coa-feldman-es1024i-fall-2022/}.
I recommend doing the WeBWorK part of the homework first.  This will
enable you to benefit WeBWorK's instant feedback before you do part
two.\\ 


\noindent {\bf Part 2: Non-WeBWorK problems}.  Here are some
instructions for how to submit this part of the assignment.
\begin{itemize}
  \setlength{\itemsep}{0mm}
\item Do the problems by hand using pencil (or pen) and paper.
  There is no need to type this assignment.
\item If you like working on a tablet, go for it. 
\item Make a pdf scan of your work using genius scan or some
  similar scanning app.  Please make the homework into a single
  pdf, not multiple pdfs.
\item Submit the assignment on google classroom.  Please don't
  email it to me.  (Between my two classes I will be receiving
  around 60 assignments a week.  Keeping track of them all in email 
  is challenging.)
\item If you want, you can do the non-WeBWorK problems in pairs and
  submit only one assignment for the two of you. \\
\end{itemize}

\noindent Here are some non-WeBWorK problems.

\begin{enumerate}
\setlength{\itemsep}{8mm}

\item The amount of caffeine in the one's body is described by the
  following function:
  \begin{equation}
    f(t) = Q_0 e^{-0.17t} \;,
    \label{eq:e}
  \end{equation}
  where $Q_0$ is the amount of caffeine in the body at time $t=0$, and
  time is measured in hours.
  \begin{enumerate}
  \item Make a graph of the function.  For your graph, assume that
    $Q_0 = 100$ mg.
  \item Use your graph to estimate the half-life of caffeine in your
    body.
  \item Re-write Eq.~(\ref{eq:e}) in the following form:
    \begin{equation}
      f(t) \, = \, Q_0 a^t \;.
    \end{equation}
  \item Use logarithms to solve exactly for the half-life of caffeine.
  \end{enumerate}

\item Air pressure $P$ decreases exponentially with height $h$, where
  $h$ is the height above sea level.  The pressure is given by the
  following function:
  \begin{equation}
    P \, = \, p_0 e^{-0.00012h} \;,
  \end{equation}
  where $h$ is measured in meters above sea level, and $P_0$ is the
  pressure at sea level.
  \begin{enumerate}
  \item What is the air pressure, as a percent of the pressure at sea 
    level, on the top of Mount Kilimanjaro?
  \item What is the air pressure, as a percent of the pressure at sea 
    level, on the top of Wheeler Peak, in New Mexico, USA?
  \item What is the air pressure, as a percent of the pressure at sea 
    level, on the top of Cadillac Mountain?
  \end{enumerate}
  
\item In an electrical outlet in the US, the voltage $V$ (in volts) is
  given as a function of time $t$ (in seconds) by the formula:
  \begin{equation}
    V(t) \, = \, V_0 \sin(120\pi t) \;.
    \label{eq:V}
  \end{equation}
  \begin{enumerate}
  \item What does $V_0$ represent in terms of voltage?
  \item What is the period of this function?
  \item How many oscillations are there in one second?
  \item In Europe, most of Asia, Africa, and the Middle East
    electricity oscillates $50$ times in one second. Write down a
    function that describes $V(t)$ describes this electricity. That
    is, write down a function similar to Eq.~(\ref{eq:V}), but which
    oscillates $50$ times in one second.
  \end{enumerate}

  
\end{enumerate}

\end{document}

  \setlength{\itemsep}{-1mm}
  \item Determine an equation for the linear function that generates
    the values in the table below.  

\begin{center}
\begin{tabular}{|| l | l ||}
\hline $x$ & $f(x)$ \\
\hline
5.2 & 27.8 \\
5.3 & 29.2 \\
5.4 & 30.6 \\
5.5 & 32.0 \\
5.6 & 33.4 \\
\hline
\end{tabular}
\end{center}

\item The graph of Fahrenheit temperature, F, as a function of Celsius
  temperature, C, is a line.  We know that 212 F and 100 C are the
  same; this is the temperature at which water boils (under standard
  pressure).  And we also know that 32 F and 0 C are the same; this is
  the temperature at which water freezes.
  \begin{enumerate}
  \item What is the slope of the graph?
  \item What is the equation of the line?  (I.e., F as a function of
    C.)
  \item Use the equation to find what Fahrenheit temperature
    corresponds to 20 C.
  \item What temperature is the same number of degrees in both C and
    F?
  \end{enumerate}

  \item The cost of planting a crop is usually a function of the
    number of acres sown. The cost of the equipment is a \emph{fixed
      cost}, because it must be paid regardless of the number of
    acres planted. The cost of supplies and labor varies with the
    number of acres planted and is called the \emph{variable cost}.
    Supposed the fixed costs of \$10,000 and the variable costs are
    \$200 per acre. Let $C$ represent the total cost of planting $x$
    acres of a crop.
    \begin{enumerate}
    \item Find a formula for $C$ as a function of $x$.
    \item Graph the function.
    \item Which feature on the graph (slope or y-intercept) represents
      the fixed cost?  Which represents the variable cost?
    \end{enumerate}

\end{enumerate}




\end{document}
