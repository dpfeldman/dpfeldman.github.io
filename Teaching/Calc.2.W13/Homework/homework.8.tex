\documentclass[12pt]{article}
\oddsidemargin=0.0in
\evensidemargin=0.0in
\textwidth=6.5in
\topmargin=-.35in
\textheight=8.4in
\usepackage{url}

\begin{document}
\pagestyle{empty}

\begin{center}
{\large {\bf Calculus II}}\\
\medskip
{\large {\bf Homework Eight}}\\
\medskip
{ {\bf Due March 1, February, 2013}}\\
\end{center}

\hspace{2mm}\\



\noindent {\bf Chapter 8.7:}

\begin{enumerate}
\setlength{\itemsep}{-1mm}
  \item 1--3
  \item 11
  \item 12
  \item 17
%  \item 20 
\end{enumerate}

\noindent {\bf Chapter 8.8:}
\begin{enumerate}
\setlength{\itemsep}{-1mm}
  \item 6
%  \item 7
%  \item 10
\end{enumerate}

\noindent {\bf Normal Distributions}

\begin{enumerate}
\setlength{\itemsep}{-1mm}
\item The height of giraffes is distributed according to a normal
  distribution with a mean of $5.2$ and a standard deviation of
  $0.3$.  
\begin{enumerate}
\setlength{\itemsep}{-1mm}
\item What fraction of giraffes are less than 4 meters tall?
\item What fraction of giraffes are between 5 and 6 meters tall?
\item What fraction of giraffes are more than 5.5 meters tall?
\end{enumerate}
Answer these questions two ways:
\begin{itemize}
\setlength{\itemsep}{-1mm}
\item Using WolframAlpha
\item Converting to z and using a z-table. See, e.g.,
  \url{http://en.wikipedia.org/wiki/Standard_normal_table}.  
\end{itemize}

\item Sarah Luke is interested in the heights of COA students compared
  to Hampshire students.  A careful study reveals that COA students
  have an average height of 63 inches and a standard deviation of 4
  inches.  Sarah then sends a team of RAs on a trip to Massachusetts
  to measure the heights of some Hampshire students.  The RA team
  manages to convince 25 Hampshire students to be measured.  The mean
  of these 25 Hampshire students is 67 inches. 
  %The standard deviation of this sample of Hampshire students is 3.
\begin{enumerate}
\setlength{\itemsep}{-1mm}
  \item What is the null hypothesis?
  \item What is the p-value?
  \item Should you reject the null?  Do you think it is likely that
    the average heights of Hampshire and and COA students are the
    same? 
\end{enumerate}

\end{enumerate}

%\noindent {\bf Chapter 8.6:}
%
%\begin{enumerate}
%\setlength{\itemsep}{-1mm}
%  \item 7
%  \item 12
%  \item A certain set of solar cells will produce 20 kWh of energy a
%    day.  In Maine one kWh is worth about 17 cents.
%\begin{enumerate}
%  \item What is the present value of the income stream provided by the
%    solar cells?  Assume that the solar cells will operate for 15
%    years. 
%  \item What is the present value of the income stream if the solar
%    cells operate forever?
%  \item {\bf Optional:}  What is the internal rate of return of the
%    solar cells, assuming a fifteen year lifetime and that the solar
%    cells cost \$10,000?
%\end{enumerate}
%
%\end{enumerate}



%\noindent {\bf Chapter 9.1:}
%
%\begin{enumerate}
%\setlength{\itemsep}{-1mm}
%\item 2
%\item 4
%\item 6
%\item 8
%\item 10
%\item 14
%\item 56
%\end{enumerate}



\end{document}
