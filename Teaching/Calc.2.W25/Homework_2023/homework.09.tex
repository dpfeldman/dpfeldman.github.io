\documentclass[12pt]{article}
\oddsidemargin=-0.2in
\evensidemargin=0.0in
\textwidth=7in
\topmargin=-0.55in
\textheight=9.4in
\usepackage{hyperref}
 
\begin{document}
\pagestyle{empty}
 
\begin{center}
{\LARGE {\bf Homework Nine}}\\
\bigskip
{\Large {\bf Calculus II}}\\
\bigskip
{\Large {\bf College of the Atlantic}}\\
\bigskip
{ {\bf Due Friday, March 10, 2023}}\\ 
\end{center}
\medskip


\noindent There are two parts to this assignment!\\

\noindent {\bf Part 1: WeBWorK}.  Do Homework 09 which you
will find the WeBWorK page here:
\url{https://webwork-hosting.runestone.academy/webwork2/coa-feldman-es3012m-winter2023}.

%I recommend doing the WeBWorK part of the homework first.  This will
%enable you to benefit WeBWorK's instant feedback before you do part
%two.\\ 


\noindent {\bf Part 2: Non-WeBWorK problems}.  Here are some
instructions for how to submit this part of the assignment.
\begin{itemize}
\item Do the problems by hand using pencil (or pen) and paper.
  There is no need to type this assignment.
\item If you like working on a tablet, go for it. 
\item Make a pdf scan of your work using genius scan or some
  similar scanning app.  Please make the homework into a single
  pdf, not multiple pdfs.
\item Submit the assignment on google classroom.  Please don't
  email it to me. Thanks.  %(Between my two classes I will be receiving
  %around 60 assignments a week.  Keeping track of them all in email 
  %is challenging.)
\item If you want, you can do the non-WeBWorK problems in pairs and
  submit only one assignment for the two of you. \\
\end{itemize}

\noindent Here are some non-WeBWorK problems.

\begin{enumerate}

  \item Suppose in an effort to stimulate the economy, Governor Mills
    gives everyone in Maine \$600.  Let's assume that there are one
    million people in Maine. So the total expense to the state of
    Maine is 600 million dollars. Of this extra money, assume that
    people spend 80\% of it and save the rest. Thus, the inital boost
    to the economy, in terms of new spending, is 80\% of 600 million
    dollars, or 480 million.

    But this extra spending is now extra income for someone
    else. Assume that of this extra income, 80\% is spent and 20\% is
    saved. This spending then is someone\footnote{We assume
     that all spending stays in Maine.} else's extra income, of which
    80\% is spent and 20\% is saved, and so on. Calculate the total
    additional spending in Maine created by the governor's \$600
    payment to all Mainers. 

\item Use the ratio test to determine whether or not the following
  series converges:
  \begin{equation}
    \sum_{n=1}^\infty \frac{(n!)^2}{(2n)!}
    \end{equation}
  Show your steps.
  
\item You and your friend live on the same road, 20 miles from each
  other. One day you both decide to leave your houses at exactly noon
  and bike toward each other on the road that connects your
  houses. You both bike at a constant speed of 10 miles/hour. Your
  friend has a parrot. The parrot leaves your friend's house at the
  same time as your friend. She flies down the road at a speed of 15
  miles/hour. When the parrot reaches you, she turns around and flies
  back to your friend. The upon reaching your friend, she turns around
  and flies to you, and so on. This continues until you and your
  friend meet, halfway between your houses. 
  \begin{enumerate}
      \item How far does the parrot fly in the first part of her
        journey: from your friend's house to you?
      \item How far does the parrot fly in the second part of her
        journey: from you to your friend?
      \item How far does the parrot fly in the third part of her
        journey: from your friend to you?
      \item Write the total distance traveled by the parrot as an
        infinite sum, and then calculate the value of the sum, to
        figure out the total distance flown by the parrot.
      \item Find another, more direct way, to determine how far the
        parrot has flown. (Hint: At what time do you and your friend
        meet.)
    \end{enumerate}
  
\end{enumerate}
    
%\begin{enumerate}
%\setlength{\itemsep}{-1mm}
%  \item Determine an equation for the linear function that generates
%    the values in the table below.  

%\begin{center}
%\begin{tabular}{|| l | l ||}
%\hline $x$ & $f(x)$ \\
%\hline
%5.2 & 27.8 \\
%5.3 & 29.2 \\
%5.4 & 30.6 \\
%5.5 & 32.0 \\
%5.6 & 33.4 \\
%\hline
%\end{tabular}
%\end{center}

%\item The graph of Fahrenheit temperature, F, as a function of Celsius
%  temperature, C, is a line.  We know that 212 F and 100 C are the
%  same; this is the temperature at which water boils (under standard
%  pressure).  And we also know that 32 F and 0 C are the same; this is
%  the temperature at which water freezes.
%  \begin{enumerate}
%  \item What is the slope of the graph?
%  \item What is the equation of the line?  (I.e., F as a function of
%    C.)
%  \item Use the equation to find what Fahrenheit temperature
%    corresponds to 20 C.
%  \item What temperature is the same number of degrees in both C and
%    F?
%  \end{enumerate}
%
%  \item The cost of planting a crop is usually a function of the
%    number of acres sown. The cost of the equipment is a \emph{fixed
%      cost}, because it must be paid regardless of the number of
%    acres planted. The cost of supplies and labor varies with the
%    number of acres planted and is called the \emph{variable cost}.
%    Supposed the fixed costs of \$10,000 and the variable costs are
%    \$200 per acre. Let $C$ represent the total cost of planting $x$
%    acres of a crop.
%    \begin{enumerate}
%    \item Find a formula for $C$ as a function of $x$.
%    \item Graph the function.
%    \item Which feature on the graph (slope or y-intercept) represents
%      the fixed cost?  Which represents the variable cost?
%    \end{enumerate}

%\end{enumerate}




\end{document}
