
\documentclass[12pt]{article}
\oddsidemargin=0.0in
\evensidemargin=0.0in
\textwidth=6.5in
\topmargin=-.45in
\textheight=8.4in
%\usepackage{doublespace}
 
\begin{document}
\pagestyle{empty}
 
\begin{center}
{\large {\bf Thermodynamics}}\\
\medskip
{\large {\bf Homework Seven}}\\
\medskip
{ {\bf Due Friday November 8, 2013}}\\
\end{center}

\noindent This assignment will be a bit longer than usual, since it
covers 1.5 weeks instead of one.  However, I think these problems will
be much less mathematically demanding than those on the last problem
set, and will involve a good bit more interesting physics.

\begin{enumerate}

  \item Problem 4.18  Optional.  This problem involves a modest amount
    of algebra and could be a good review of the physics of adiabatic
    and isothermal processes. 
  \item Problem 4.20  Optional.  This problem involves a less modest
    amount of algebra.  I'm not sure the process of doing this problem
    will lead to deep understanding, but the final result---the
    efficiency of a Diesel engine---is a useful and important result,
    although it is not a simple formula.  
 
  \item Problem 4.24  
  \item Problem 4.26
  \item Problem 5.4
  \item Problem 5.6
  \item Problem 5.11
  \item Problem 5.20

\end{enumerate}


\end{document}
