
\documentclass[12pt]{article} %other options include letter and book
% One can also make presentation slides using the 'beamer' class
\usepackage[english]{babel}

% Set page size and margins
% Replace `letterpaper' with `a4paper' for UK/EU standard size
\usepackage[letterpaper,top=2cm,bottom=2cm,left=3cm,right=3cm,marginparwidth=1.75cm]{geometry}

% Useful packages
\usepackage{amsmath,amssymb,latexsym} % gives some additional math symbols and environments
\usepackage{graphicx} % for including graphics
\usepackage[colorlinks=true, allcolors=blue]{hyperref}

% After all the settings and usepackage commands, begin the document

\begin{document}

\noindent Here are some problems for Wednesday/Thursday May 23/24,
2023. Aim to share solutions to one or two of the problems. 
\bigskip

\begin{enumerate}
\setlength{\itemsep}{4mm} % increases spacing between list items

\item Prove that there do not exist three distinct numbers $a,b$, and
  $c$ for which:
  $a+b+c$, $ab$, $ac$, $bc$, and $abc$ are all equal.

  \item Prove that if $x$ and $y$ are positive real numbers, then
    $x+y \geq 2 \sqrt{xy}$.   

\item Prove that there does not exist an $n \in \mathbb{N}$ for which
  $n^2 + n + 1$ is a perfect square.

\item Prove that if $x$ and $y$ are irrational, then $x+y$ is also
  irrational.

\item Prove that if $x$ is irrational, then $x+y$ is also
  irrational.

\item Prove that $\forall x, y \in \mathbb{R}$, such that $x \neq
  y$,
  \begin{equation}
    \frac{x}{y} + \frac{y}{x} \, \leq \, 2 \;.
  \end{equation}


\item For all real numbers $x$, with $0<x<1$, prove that
  \begin{equation}
    \frac{1}{x(1-x)} \leq 4 \;.
  \end{equation}

  \item Let $a,b,c$ be positive real numbers. Prove that if $ab = c$,
    then $a \leq \sqrt{c}$ or $b \leq \sqrt{c}$. 

\item Suppose $n \in \mathbb{Z}$ is a composite integer. Then $n$ has
  a prime divisor less than or equal to $\sqrt{n}$. 
    
\item Prove that $x \in \mathbb{R}$ is irrational if and only if it
  has a different distance from each rational number.

\item Prove that $S_n \notin \mathbb{Z}$ for all $n\geq 2$, where
  \begin{equation}
    S_n \, = \, \sum_{k=1}^n \frac{1}{k} \;.
  \end{equation}
 

\end{enumerate}


The last three problems look interesting and potentially
challenging. If more than one group wants to give these a try, that's
fine. 



\end{document}
