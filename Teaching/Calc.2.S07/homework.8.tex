\documentclass[12pt]{article}
\oddsidemargin=0.0in
\evensidemargin=0.0in
\textwidth=6.5in
%\topmargin=-.75in
%\textheight=9in
%\usepackage{doublespace}
 
\begin{document}
\pagestyle{empty}
 
\begin{center}
{\large {\bf Calculus II}}\\
\medskip
{\large {\bf Homework Eight}}\\
\medskip
{ {\bf Due Friday 18 May, 2007}}\\
\end{center}

\hspace{2mm}\\

\noindent {\bf Chapter 9.1:}

\begin{enumerate}
\setlength{\itemsep}{-1mm}
\item 8
\item 9
\item 13
\item 14
\item 20-29
\item 48
\item 56
\end{enumerate}

\noindent {\bf Chapter 9.2:}

\begin{enumerate}
\setlength{\itemsep}{-1mm}
\item 18-21
\item 30
\item 31
\item Two trains are 100 km apart.  At noon, they start moving
  directly toward each other at a speed of 10 m/s.  Eventually, they
  crash.  Initially, a bird is sitting on one of the trains.  At noon,
  the bird flies directly toward the other train at 20m/s.  As soon as
  it reaches the train, it turns around and heads to the other train.
  As soon as it reaches this train, it again turns around and heads
  toward the other one, and so on, traveling at 20 m/s all the while.
  The bird zigs and zags back and forth between the two trains.
  Eventually, when the trains crash, the bird is crushed, too.  How
  far did the bird travel?
\begin{enumerate}
    \item Do this problem the easy way.
    \item Do this problem the hard way, by expressing the distance
    traveled as a geometric series and summing the series.
\end{enumerate}


\end{enumerate}

\noindent {\bf Chapter 9.3:}

\begin{enumerate}
\setlength{\itemsep}{-1mm}
\item 10-16, even only.  Be sure to explain your reasoning
\end{enumerate}




\end{document}
