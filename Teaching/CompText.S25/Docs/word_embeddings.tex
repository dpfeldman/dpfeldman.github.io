\documentclass[12pt]{article}
\oddsidemargin=-0.25in
\evensidemargin=-0.25in
\textwidth=7.0in
\topmargin=-.75in
\textheight=9.4in
\usepackage{graphicx}
\usepackage{url}

%\renewcommand{\arraystretch}{1.3}

\begin{document}
\pagestyle{empty}
\begin{center}
{\Large {\bf Word Embedding Activity}}\\
\medskip
{\large {\bf Language, Power, Computation}}\\
\medskip
{\bf College of the Atlantic. April 14, 2025}\\
\end{center}

\vspace{-7mm}

\begin{figure}[h]
\begin{center}
\vspace{1mm}
\includegraphics[width=2.5in]{unicorn2.png}
\vspace{-22mm}
\caption{A unicorn. Just because. You can color it in if you want.
  Image from \protect\url{https://freesvg.org/unicorn-for-coloring}} 
%\label{fig:speed}
\end{center}
\end{figure}

\vspace{-10mm}

\begin{enumerate}

\item Grab some sticks and a glue gun.  Maybe grab some cardboard,
  too, so the glue doesn't get on the table.

\item Use the glue and sticks to make a coordinate system.

\item Using duct tape and markers (or other methods if you wish),
  label the three axes: valence, arousal, dominance, in that
  order. Choose a right-handed coordinate system.

\item We will then give you some words. Give each word a score between
  $0$ and $1$ for the following categories:
  \begin{itemize}
    \item {\bf Valence}. How positive or
      pleasurable the word\footnote{I.e., the meaning or meanings of
        the word} is
    \item {\bf Arousal}. How excited or active the word is.
    \item {\bf Dominance}. How powerful or controlling the word is.
  \end{itemize}
  Debate and ponder for a bit. Obviously, there is no right answer.

\item Use a stick to plot each of your words as a vector in the
  coordinate system you made. Label your words. 
  
\item Which of your words are close together? How would you measure
  this?

\item What issues/questions/concerns/dilemmas did you encounter while
  doing this exercise? 

%\begin{center}
%\begin{tabular}{|| l | c ||}
%  \hline
%  $x$ & $u(x)$ \\
%  \hline
%  0 &  30 \\
%  2 & 40 \\
%  4 & 55\\
%  6 & 55\\
%  8 & 60\\
%  10 & 70\\
%  12 & 75\\
%  \hline 
%\end{tabular}
%\end{center}


\end{enumerate}


\end{document}


