\documentclass[12pt]{article}
\oddsidemargin=0.0in
\evensidemargin=0.0in
\textwidth=6.5in
\topmargin=-0.35in
\textheight=9in
%\usepackage{doublespace}
 
\begin{document}
\pagestyle{empty}
 
\begin{center}
{\LARGE {\bf Homework Three}}\\
\bigskip
{\Large {\bf Physics I}}\\
\bigskip
{\Large {\bf College of the Atlantic}}\\
\bigskip
{ {\bf Due Friday, October 1, 2021}}\\ 
\end{center}
\medskip

\noindent There are two parts to this assignment.\\

\hspace{2mm}\\

\noindent {\bf Part 1: Edfinity}.  Do Homework 03 which you will find
on your edfinity account.  Note that there are two separate parts on
edfinity for Homework 03: one for Chapter C04 and one for Chapter
C05.  I recommend doing the edfinity part of the homework first.  This
will enable you to benefit edfinity's instant feedback before you do
part two.\\ 


\noindent {\bf Part 2: Problems from the Textbook}.  Here are some
instructions for how to submit this part of the assignment.
\begin{itemize}
\item Do the problems by hand using pencil (or pen) and paper.
  There is no need to type of this assignment.
\item Make a pdf scan of your work using genius scan or some
  similar scanning app.  Please make the homework into a single
  pdf, not multiple pdfs.
\item Submit the assignment on google classroom.  Please don't
  email it to me.  (Between my two classes I will be receiving
  over 45 assignments a week.  Keeping track of them all in email
  is challenging.)\\
%\item Do problem 3.5 in pairs. Work with someone else in the class
%  and hand in only one solution for the two of you.
%\item If you want, you can do all the problems in pairs and hand
%  in only one set of solutions.\\
\end{itemize}

\noindent The textbook problems are:   

\begin{enumerate}
\setlength{\itemsep}{-1mm}
\item {\bf Chapter C4:}
  \begin{enumerate}
  \item C4S.2
  \item C4S.5 (The idea here is that the explosion due to the missile
    can be viewed as an internal interaction. Thus, the motion of the
    center of mass of the asteroid remains unchanged.)
  \end{enumerate}

\item {\bf Chapter C5:}
  \begin{enumerate}
  \item C5S.5
  \item C5S.6
  \item C5S.9
  \end{enumerate}
\end{enumerate}







\end{document}
