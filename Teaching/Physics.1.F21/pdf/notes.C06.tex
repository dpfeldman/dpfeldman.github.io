\documentclass[12pt]{article}
\oddsidemargin=-0.250in
\evensidemargin=0.0in
\textwidth=6.75in
\topmargin=-.35in
\textheight=9in
\usepackage{graphicx}
 

\begin{document}
\pagestyle{empty}

\begin{center}
{\Large {\bf Chapter C6: Introduction to Energy}}\\
\medskip
{\large {Physics I}}\\
\medskip
College of the Atlantic\\
\end{center}
\medskip

\noindent {\large {\bf C6.1:  Interactions and Energy}}

\begin{itemize}
\item %Recall the story about Jane and Anna and counting blocks.  
There is some scalar quantity (energy) that remains constant no matter
what.  Energy is not ``seeable'' like momentum is; energy is some
number that has to be calculated, and there is more than one formula
for energy. 
\item There are, for now, two sorts of energy, {\em kinetic and
potential}.
\begin{enumerate}
    \item {\em Kinetic:}  This is a property of a single moving object.  
    \item {\em Potential:} This arises from an interaction between two
    objects.  It is not the property of a single object.
\end{enumerate}
\item The total energy of a system is the sum of all the potential and
kinetic energies.  This is what Eq.~(C6.2) says.\\
\end{itemize}


\noindent {\large {\bf C6.2:  Kinetic Energy}}
\begin{itemize}

\item The kinetic energy $K$ of an object of mass $m$ moving with
speed $v$ has a kinetic energy given by 
\begin{equation}
K \, \equiv \, \frac{1}{2}mv^2 \; .
\end{equation}  
For now, we should view this as a definition. 
\item The units for energy are {\em Joules}:
\begin{equation}
1 {\rm J} \, \equiv \, \frac{ {\rm kg \; m}^{2}}{ {\rm s}^2} \;.
\end{equation}
\item The last several paragraphs of this section argue that because
  the earth is so massive, interactions change its velocity by a
  minuscule amount, and hence we can ignore the earth's kinetic
  energy. \\

\end{itemize}
\newpage
\noindent{\large {\bf C6.3:  Measuring Potential Energy}}

\begin{itemize}
\item Potential energy is measure of the extent to which an
interaction can give an object kinetic energy.  By measuring the
kinetic energy an object gets, we can measure the potential energy
associated with the interaction.
\item Carrying out this procedure, we are led to:
\begin{equation}
	V(r_i) - V(r_f) = mg(z_i - z_f) \;.
\end{equation}
This equation is a bit of a mess notationally.

\item A simpler way to write this is Eq.~(C6.12):  Gravitational
Potential energy of an object of mass $m$ a height $z$ above a
reference position ($z=0$) is given by 
\begin{equation}
V(z) \equiv mgz \;.
\end{equation}


\end{itemize}


\noindent {\large {\bf C6.4:  Negative Energy?}}
\begin{itemize}
\item Potential energy can be negative.  All that matters physically
are potential energy {\em differences}. 
\item  When doing problems with gravitational potential energy, always
state your reference level ($z=0$).  
\item If you use a positive value for $g$, then up must be positive
for $z$. \\
\end{itemize}

\noindent {\large {\bf C6.5:  A Look Ahead}}
\begin{itemize}
\item Since all that matters is energy difference, Moore always
writes conservation of energy in difference form:
\begin{eqnarray}
\label{eq:E}  0 \, &=& \, \Delta E \;, \\
              0 \, &=& \, \Delta K_1 + \Delta K_2 + \Delta V \;.  
\end{eqnarray}
\item Sometimes there is hidden energy that cannot be accounted for
  with $V$ or $K$.  The change in this hidden energy is, for now,
  denoted $\Delta U$:
\begin{equation}
    0 \, = \, \Delta K_1 + \Delta K_2 + \Delta V + \Delta U\;.  
\end{equation}
\item I usually write Eq.~(\ref{eq:E}) as:
\begin{equation}
  E_i \, = \, E_f \;.
\end{equation}
\end{itemize}



% **********************************************************************



\end{document}


