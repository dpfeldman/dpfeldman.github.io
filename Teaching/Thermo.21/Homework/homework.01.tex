\documentclass[12pt]{article}
\oddsidemargin=0.0in
\evensidemargin=0.0in
\textwidth=6.5in
\topmargin=-0.35in
\textheight=9in
\usepackage[utf8]{inputenc}
\usepackage[russian]{babel}
%\usepackage{doublespace}
 
\begin{document}
\pagestyle{empty}
 
\begin{center}
{\LARGE {\bf Homework One}}\\
\bigskip
{\Large {\bf Thermodynamics}}\\
\bigskip
{\Large {\bf College of the Atlantic}}\\
\bigskip
{ {\bf Due Friday, April 2, 2021}}\\ 
\end{center}
\medskip


\noindent There are two parts to this assignment.\\

\noindent {\bf Part 1: Short Reflection}.  There is prompt on google
classroom that I'd like you to write a short response to.\\

\noindent {\bf Part 2: Problems from the Textbook}.  Here are some
instructions for how to submit this part of the assignment.
\begin{itemize}
  \setlength{\itemsep}{1mm}
\item Do the problems by hand using pencil (or pen) and paper.
  There is no need to type up this assignment.
\item Make a pdf scan of your work using genius scan or some
  similar scanning app.  Please make the homework into a single
  pdf, not multiple pdfs.
\item Submit the assignment on google classroom.  Please don't
  email it to me. 
\item Do the last problem in the groups you were in on the first day
  of class.  Hand in only one write-up for your group.
\item If you want, you can do other problems in your group and hand
  in only one set of solutions for those problems, too.
\end{itemize}


\begin{enumerate}
\setlength{\itemsep}{-1mm}
  \item 1.4
  \item 1.9
  \item 1.11
  \item 1.14
  \item This is a problem based on question 1.16 from the textbook.
    The goal is to use Newton's second law ($\vec{F}_{\rm net} =
    m\vec{a}$) and the ideal gas law to derive the barometric
    equation.  To do so, consider a slab of air with a thickness of
    $\Delta z$ at rest at a height $z$ above the surface of the earth.
    Denote by $M$ the mass of the air in the slab.  Let $A$ be the
    horizontal area of the slab.  
    \begin{enumerate}
    \item Use Newton's law to derive an expression for
      $\frac{dP}{dz}$, the rate at which pressure changes with
      altitude.  \emph{Hints:}
      \begin{itemize}
      \item The derivative is defined as:
        \begin{equation}
          \frac{dP}{dz} \, = \, \lim_{\Delta z \rightarrow 0}
          \frac{P(z+\Delta z) - P(z)}{\Delta z} \;.
        \end{equation}
      \item There are three forces acting on the slab.
      \end{itemize}
      \item Use your answer to the previous problem and the ideal gas
        law to show that:
        \begin{equation}
          \frac{dP}{dz} \, = \, -\frac{mg}{kT}P \;,
          \label{eq:barometric}
          \end{equation}
        where $m$ is the average mass of the air molecules.  This
        equation is known as the barometric equation.
      \item Show that, assuming that $T$ is constant, the solution to
        Eq.~(\ref{eq:barometric}) is given by:
        \begin{equation}
          P(z) \, = \, P(0) e^{-mgz/kT} \;,
          \label{eq:solution}
          \end{equation}
        where $P(0)$ is the pressure at sea level.
      \item Use Eq.~(\ref{eq:solution}) to calculate the pressure, in
        atmospheres, at the following locations:
        \begin{enumerate}
        \item Cadillac Mountain
        \item Katahdin Mountain
        \item Cerro El Pital
        %\item Mount Elbrus
        \item Гора Эльбрус
        \end{enumerate}
        Assume that the pressure at sea level is $1$ atm. 
    \end{enumerate}

\end{enumerate}






\end{document}
