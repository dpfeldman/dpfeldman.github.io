\documentclass[12pt]{article}
\oddsidemargin=0.0in
\evensidemargin=0.0in
\textwidth=6.5in
\topmargin=-0.35in
\textheight=9in
\usepackage[utf8]{inputenc}
%\usepackage[russian]{babel}
\usepackage[normalem]{ulem}
%\usepackage{doublespace}
 
\begin{document}
\pagestyle{empty}
 
\begin{center}
{\LARGE {\bf Homework Four}}\\
\bigskip
{\Large {\bf Thermodynamics}}\\
\bigskip
{\Large {\bf College of the Atlantic}}\\
\bigskip
{ {\bf Due Friday, April 23, 2021}}\\ 
\end{center}
\medskip


\noindent There is one part to this assignment.\\

%\noindent {\bf Part 1: Short Reflection}.  There is prompt on google
%classroom that I'd like you to write a short response to.\\

\noindent {\bf Part 1: Problems from the Textbook}.  Here are some
instructions for how to submit this part of the assignment.
\begin{itemize}
  \setlength{\itemsep}{1mm}
\item Do the problems by hand using pencil (or pen) and paper.
  There is no need to type up this assignment.
\item Make a pdf scan of your work using genius scan or some
  similar scanning app.  Please make the homework into a single
  pdf, not multiple pdfs.
\item Submit the assignment on google classroom.  Please don't
  email it to me. 
\item If you want to do one or more of these problems one or two other
  people and hand in only one write-up, go for it. 
%\item Do the last problem in the groups you were in on the first day
%  of class.  Hand in only one write-up for your group.
%\item If you want, you can do other problems in your group and hand
%  in only one set of solutions for those problems, too.
\end{itemize}


\begin{enumerate}
\setlength{\itemsep}{-1mm}
\item 2.2
\item 2.5 (a, b, and c)
\item 2.7
\item 2.8

\item Deriving a useful approximation.
  \begin{enumerate}
  \item Derive the approximation
    \begin{equation}
      \ln(1+x) \, \approx x \;,
      \label{eq:approx}
    \end{equation}
    which is valid for $|x|\ll 1$.  To do so, figure out the equation
    of the line tangent to $\ln(1+x)$ at $x=0$.
    \item Check the accuracy of the approximation in
      Eq.~(\ref{eq:approx}) for $x=0.1$, $x=0.01$, and $x=0.001$.
  \end{enumerate}

  
\item 2.16

\end{enumerate}






\end{document}
