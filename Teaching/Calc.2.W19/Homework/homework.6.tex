\documentclass[12pt]{article}
\oddsidemargin=0.0in
\evensidemargin=0.0in
\textwidth=6.5in
%\topmargin=-.75in
%\textheight=9in
\usepackage{url}
\usepackage{hyperref}
 
\begin{document}
\pagestyle{empty}
 
\begin{center}
{\large {\bf Calculus II}}\\
\medskip
{\large {\bf Homework Six}}\\
\medskip
{ {\bf Due Friday, 15 February, 2019}}\\
\end{center}



\noindent {\bf Chapter 8.1:}  
\begin{itemize}
\setlength{\itemsep}{0mm}
\item WeBWorK Assignment: None
\item Textbook Problems:
  \begin{enumerate}
  \setlength{\itemsep}{-1mm}
    \item 2
    \item 4
    \item 12
    \item 24
    \item 25 (Optional)
  \end{enumerate}
\end{itemize}

\noindent {\bf Chapter 8.2:}  
\begin{itemize}
\setlength{\itemsep}{0mm}
\item WeBWorK Assignment: None
\item Textbook Problems:
  \begin{enumerate}
  \setlength{\itemsep}{-1mm}
    \item 2
    \item 10
    \item 11
    \item 12
    \item 29
  \end{enumerate}
\end{itemize}


%\noindent {\bf Chapter 8.6:}  
%\begin{itemize}
%\setlength{\itemsep}{0mm}
%\item WeBWorK Assignment: None
%\item Textbook Problems:
%  \begin{enumerate}
%  \setlength{\itemsep}{-1mm}
%    \item 9
%    \item In Moore, Michael R. ``Native American water rights:
%      Efficiency and fairness.'' \emph{Nat. Resources J}. 29 (1989):
%      763, the author discusses attempts to quantify the present value
%      of the water rights that Native American tribes were denied
%      access to.  For example, suppose that yearly value of an
%      acre-foot of water is \$5.  (This value arises because the water
%      can be used to increase agricultural productivity.)  Suppose a
%      tribe had been denied access to this water for the past $100$
%      years.  What is the value of this water today?  I.e., what is
%      the total value of the foregone profits over the last $100$
%      years?  
%    \begin{enumerate}
%        \item Using a discount rate of $3$\% write an integral that
%          gives an expression for the present value of $100$ years of
%          foregone profits.  Briefly explain how you arrived at this
%          integral. 
%        \item Evaluate the expression you arrived at and compare the
%          answer to that obtained in footnote 38 on page 774 of
%          Moore's paper.  A pdf can be found at:
 % \url{lawschool.unm.edu/nrj/volumes/29/3/07_moore_native.pdf}. 
 %       \item What would be the value of the water if one had not
 %         applied discounting?  
 %   \end{enumerate}
 % \item You decide to start a kombucha business.  Doing to will
 %   require a purchase of \$100,000 in kombucha equipment.  You expect
 %   to sell 40,000 bottles of kombucha each year, making a profit of
 %   fifty cents on each bottle sold.  The kombucha equipment will last
 %   for ten years; after that it will become so gross that it will
 %   need to be replaced. 
%\begin{enumerate}
%     \item What is the present value of the ten-year income stream of
%       profits from your kombucha sales?  Assume a discount rate of
%       $r=0.05$. 
%     \item What is the present value of the income stream if you use a
%       discount rate of $r.0.1$.
%     \item ({\bf Optional, but recommended, especially for those
%       interested in business and/or sustainable energy.})  Note that
%       as the discount rate increases, the present value decreases.
%       How large a discount rate would you have to apply so that the
%       value of the income stream was equal to \$100,000?  To figure
%       this out, keep $r$ as variable and evaluate the integral.
%       Then set it equal to \$100,000 and solve for $r$.  You'll need
%       WolframAlpha for this last step, as the equation you'll get is
%       one that can't be solved algebraically.  (The quantity you just
%       calculated is known as the \emph{internal rate of return}
%       (IRR). The IRR is the discount rate that makes the present
%       value of an income stream equal to the cost of that income
%       stream---in this case the cost of your kombucha equipment.  The
%       IRR is a very useful and commonly used metric for evaluating
%       investments. 
%  \end{enumerate}
%  \end{enumerate}
%\end{itemize}


\end{document}
