
\documentstyle[12pt,url,epsf]{article}
\oddsidemargin=0in
\textwidth=6.5in
%\topmargin=-.25in
%\textheight=9in
%\usepackage{doublespace}

\renewcommand{\arraystretch}{1.3}

\begin{document}

\pagestyle{empty}

\begin{center}
{\large {\bf Homework assignment two}}\\
\end{center}
\bigskip
\hspace{1cm}\\


{\bf Due Friday September 27, 2002, 4:00 PM} \\

\begin{enumerate}

\item Yeomans, problem 2.1.

\item Yeomans, problem 2.2.

\item Yeomans, problem 2.3.

\item Yeomans, problem 2.4.

\item Estimate the susceptibility critical exponent $\gamma$ for the
two-dimensioal ferromagnetic Ising model as follows.

\begin{enumerate}

\item Using one of your existing Monte Carlo codes, do a run at a
bunch of temperatures near the critical temperature, calculating
$\chi$ at each temperature.  A lattice size of around $100$ by $100$
should be sufficient. 

\item Then, make a log-log plot of the susceptibility $\chi$ versus
the reduced temperature $T-T_c$.  The slope of line is the critical
exponent.  Be sure to determine error bars to go along with your
estimated slope.

\item Compare your result with the known, exact critical exponent.
Also compare you results with each other.  Discuss briefly.

\item Comment on sources of error in your calculation.  You may wish
to consult section 8.3.1 of Newman and Barkema.

\end{enumerate}

\end{enumerate}



\end{document}


