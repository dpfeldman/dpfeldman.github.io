
\documentstyle[12pt]{article}
\oddsidemargin=0in
\textwidth=6.25in
%\topmargin=-.75in
%\textheight=9in
%\usepackage{doublespace}

\renewcommand{\arraystretch}{1.3}

\begin{document}
\pagestyle{empty}

\begin{center}
{\large Lab 2:  Vectors and Air Tracks}\\
\end{center}


\begin{enumerate}

\item {\bf Trigonometry Warm Up.} 

\begin{enumerate}
\item You stand 50 meters away from a flag pole.
You have to look at an angle of 53 degrees from the horizon to see the
top of the pole.  What is the pole's height?

\item  You stand 200 meters away from a tree that's 100 meters tall.
At what angle must you tilt your head so that you look straight at the
top of the tree?

\end{enumerate}

\item {\bf Applied Trigonometry}

\begin{enumerate}

\item Grab a sextant (or two).  Go outside and figure out how to use
it.  (Read the manual and talk to me.)

\item Go to the dock and carefully measure the angle above the horizon
that the sun is.  Make a note of the time.

\item  Figure out the height of the top of the library.

\item  Measure the height of the tree of your choice.

\item Go back to the dock and measure the position of the sun again.
Make a note of the time.  Did the sun's position change more or less
than you expected?

\end{enumerate}


\item {\bf Momentum Warm Up}

\begin{enumerate}

\item Review the textbook's discussion of mass in Section C2.6.  

\end{enumerate}


\item {\bf Play with Air Tracks}

\begin{enumerate}

  \item Set up the situation used to measure mass as described in the
  text.  Try it out using at least two different test objects.
  Estimate the velocities as best you can.  (This probably won't be
  that well---don't worry about it.)  Then calculate the masses of the
  two different test objects.  What units is your answer in?

\end{enumerate}


\end{enumerate}


\end{document}
