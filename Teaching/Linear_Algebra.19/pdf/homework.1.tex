\documentclass[11pt]{article}
\oddsidemargin=0.0in
\evensidemargin=0.0in
\textwidth=6.5in
\topmargin=-.25in
\textheight=8.5in
 
\begin{document}
\pagestyle{empty}
 
\begin{center}
{\Large {\bf Linear Algebra Homework One: Corrected Version}}\\
\medskip
{\large {\bf College of the Atlantic}}\\
\medskip
{\large {\bf Due Friday January 11, 2019}}\\
\medskip
\end{center}

\noindent {\bf This assignment is complete!  I will not be adding an
  more problems.}  


\noindent Please include a cover sheet for this assignment.\\

\noindent {\bf Chapter 1.1:}
\begin{enumerate}
\setlength{\itemsep}{-1mm}
  \item 5
  \item 9
  \item 18
  \item 36
  \item Consider Newton's law of cooling
\begin{equation}
  \frac{dT}{dt} \, = \, -k(T-A) \;.
\label{eq:newton}
\end{equation}
\begin{enumerate}
    \item Show that $T(t) = A + C e^{-kt}$ is a solution to
      Eq.~(\ref{eq:newton}), where $C$ is a constant.
    \item Let the ambient temperature $A = 5$.  Find the solution
      to Eq.~(\ref{eq:newton}) that has the value of $T=40$ at $t=0$.
      (This describes how an object that is initially at $40$ degrees
      cools off if is placed outside on a brisk $5$ day.) 
    \item Let $k=0.1$, and as before $A=5$.  Using these values,
      sketch a plot of the $T(t)$ you found in the previous problem.
      It's fine to use a computer to make this plot for you, but think
      about why it has the shape it does.  Does your plot make sense
      physically? 
\end{enumerate}
  \item {\bf Optional, but recommended.}  Consider again the
    differential equation Eq.~(\ref{eq:newton}).  
\begin{enumerate}  
  \item Define a new
    function $y = T-A$, the difference between the temperature $T$ of
    the object and the ambient temperature $A$.  After plugging in you
    should get another differential equation where $y(t)$ is the
    unknown function instead of $T(t)$.
  \item Hey!  That new differential equation looks familiar.  Write
    down its general solution.
  \item Then use the $y(t)$ you just figured out to write down the
    solution $T(t)$.\\
\end{enumerate}

\end{enumerate}

\noindent {\bf Chapter 1.2:}
\begin{enumerate}
\setlength{\itemsep}{-1mm}
  \item 1
  \item 4
  \item {\bf Optional.} 43.  Not particularly
    differential-equations-ey, but perhaps amusing.
\end{enumerate}


\end{document}
