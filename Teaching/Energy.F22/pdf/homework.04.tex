\documentclass[12pt]{article}
\oddsidemargin=0.0in
\evensidemargin=0.0in
\textwidth=6.5in
\topmargin=-0.35in
\textheight=9in
\usepackage{hyperref}
 
\begin{document}
\pagestyle{empty}
 
\begin{center}
{\LARGE {\bf Homework Four}}\\
\bigskip
{\Large {\bf Physics and Math of Sustainable Energy}}\\
\bigskip
{\Large {\bf College of the Atlantic}}\\
\bigskip
{ {\bf Due Friday, October 7, 2022}}\\ 
\end{center}
\medskip


\noindent There are two parts to this assignment.\\

\noindent {\bf Part 1: WeBWorK}.  Do Homework 04 which you
will find the WeBWorK page here:
\url{https://webwork.runestone.academy/webwork2/coa-feldman-es1056i-fall-2022/}.
I recommend doing the WeBWorK part of the homework first.  This will
enable you to benefit WeBWorK's instant feedback before you do part
two.\\ 


\noindent {\bf Part 2: Problems from the Textbook}.  Here are some
instructions for how to submit this part of the assignment.
\begin{itemize}
\item Do the problems by hand using pencil (or pen) and paper.
  There is no need to type this assignment.
\item If you like working on a tablet, go for it. 
\item Make a pdf scan of your work using genius scan or some
  similar scanning app.  Please make the homework into a single
  pdf, not multiple pdfs.
\item Submit the assignment on google classroom.  Please don't
  email it to me.  (Between my two classes I will be receiving
  around 60 assignments a week.  Keeping track of them all in email 
  is challenging.)\\
\end{itemize}

\noindent The problems you should do are from {\bf Chapters 18 and
  Appendix C} of the book:  \\

\noindent {\bf Chapter 18}
\begin{enumerate}
\setlength{\itemsep}{-1mm}
\item 18.9
\item {\bf Optional}: 18-5--18.18. These problems lead you through a
  derivation of the Betz Limit. Recommended for folks who want a
  modest physics/math challenge. (Disrecommended for everyone
  else.)
  \item And here's a Chapter 18 problem that's not in the book
    yet... This problem is about the Bethel Windfarm, which is a few
    kilometers to the West of Dimmitt, Texas, USA.
    \begin{enumerate}
      \item Find the wind farm on google or bing maps, and use the
        mapping tool to estimate the area of the wind farm. I think
        bing might be easier to use than google, since the picture is
        a little clearer.  Express your area in km$^2$.
        \item What is the nameplate capacity of the Bethel Wind Farm?
          To answer this, go to this website
          \url{https://atlas.eia.gov/apps/all-energy-infrastructure-and-resources/explore},
          which is a map of basically all US energy infrastructure.
          Find the Bethel Wind Farm.  Click on the turbine icon, and
          a pop-up window will appear with two pages of info. Click to
          the second one, and you will see information about the
          windfarm.
      \item How much energy did the wind farm generate in 2021? On the
        info tab on the previous website, you will see ``View Data in
        the Electricity Data Browser'' on the top.  Click on the
        link.  You will then be sent to a page that has information
        about the electricity generated by the wind farm. 
        Do not add up the monthly totals by hand. Click on the
        ``annual'' tab and the website will do the addition for you.
      \item What is the capacity factor of the Bethel Wind Farm?
        \item What is the power density, in W/m$^2$, of the Bethel
          Wind Farm?
    \end{enumerate}
    \end{enumerate}


\noindent {\bf Appendix C}
\begin{enumerate}
  \setlength{\itemsep}{-1mm}
\item C.9
\item C.11\\ 
\end{enumerate}

\noindent Information about how to access the book and your WeBWorK
account is also included on the pinned post on google classroom. 






\end{document}
