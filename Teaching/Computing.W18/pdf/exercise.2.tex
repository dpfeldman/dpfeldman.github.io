\documentclass[12pt]{article}
\oddsidemargin=0.0in
\evensidemargin=0.0in
\textwidth=6.5in
%\topmargin=-.75in
%\textheight=9in
%\usepackage{doublespace}
 
\begin{document}
\pagestyle{empty}
 
\begin{center}
{\Large {\bf Scientific Computing}}\\
\medskip
{\large {\bf College of the Atlantic}}\\
\medskip
{\large {\bf Looping and Another Plot}}\\
\medskip
\end{center}

\hspace{2mm}\\

\noindent Here are a few things to try before class on Tuesday,
January 16. 

\hspace{2mm}\\
\hspace{2mm}\\

\begin{enumerate}
\setlength{\itemsep}{15mm}
\item Write programs that do the following:
\begin{itemize}
\setlength{\itemsep}{-1mm}
  \item Prints ``hello'' seven times
  \item Prints out the integers from 1 to 10
  \item Adds up the integers from 0 to 100 and prints the result.
    (The answer should be 5050.)
  \item Prints out numbers from 0 to 5 in increments of 0.25.  I.e.,
    the program should output $0, 0.25, 0.5, 0.75, ... 4.5, 4.75,
    5.0$. 
\end{itemize}

\item To practice plotting a bit more, make a program that produces
a nice smooth plot of the function $3x^2$ from x = 0 to 4.  Make the
plot be a solid blue line and label the axes ``x'' and ``f(x)''. 

\end{enumerate}

\hspace{2mm}\\
\hspace{2mm}\\

\noindent We'll go over these in class on Tuesday.\\

\end{document}
