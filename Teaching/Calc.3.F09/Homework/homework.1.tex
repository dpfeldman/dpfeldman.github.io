\documentclass[12pt]{article}
\oddsidemargin=0.0in
\evensidemargin=0.0in
\textwidth=6.5in
%\topmargin=-.75in
%\textheight=9in
%\usepackage{doublespace}
 
\begin{document}
\pagestyle{empty}
 
\begin{center}
{\large {\bf Calculus III}}\\
\medskip
{\large {\bf Homework One}}\\
\medskip
{ {\bf Due Friday 18 September, 2009}}\\
\end{center}

\hspace{2mm}\\

\noindent {\bf Chapter 12.1:}

\begin{enumerate}
\setlength{\itemsep}{-1mm}
  \item 1
  \item 7
  \item 8
  \item 20
  \item 22
  \item 23
  \item 24

\end{enumerate}


\noindent {\bf Chapter 12.2:} You can use WolframAlpha to check your
work if you want. But don't just mindlessly plug the formulas into
WolframAlpha. Think about why the graphs are shaped the way they are.  

\begin{enumerate}
\setlength{\itemsep}{-1mm}
  \item 2
  \item 3--5
  \item 14
  \item 16--17
  \item 19

\end{enumerate}



\noindent {\bf Chapter 12.3:}

\begin{enumerate}
\setlength{\itemsep}{-1mm}
  \item 1
  \item 7
  \item 11
  \item 17
  \item 18
  \item 21
  \item 32
  \item 33.  Why is this shape called a ``monkey saddle.''  (If you
    consult other sources for this one, be sure to site them.)

\end{enumerate}



\end{document}
