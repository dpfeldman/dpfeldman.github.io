\documentclass[12pt]{article}
\oddsidemargin=0.0in
\evensidemargin=0.0in
\textwidth=6.5in
\topmargin=-0.35in
\textheight=9in
\usepackage{hyperref}
 
\begin{document}
\pagestyle{empty}
 
\begin{center}
{\LARGE {\bf Homework Four}}\\
\bigskip
{\Large {\bf Physics and Math of Sustainable Energy}}\\
\bigskip
{\Large {\bf College of the Atlantic}}\\
\bigskip
{ {\bf Due Friday, January 29, 2021}}\\ 
\end{center}
\medskip


%\noindent There are two parts to this assignment.\\

\noindent {\bf Part 1: Edfinity}.  Do Homework 04 which you will find
on your edfinity account.  I recommend doing the edfinity part of the
homework first.  This will enable you to benefit edfinity's instant
feedback before you do part two.\\


\noindent {\bf Part 2: Problems from the Textbook/Discussion Section}.
Here are some 
instructions for how to submit this part of the assignment.
\begin{itemize}
\setlength{\itemsep}{0mm}
\item Do the problems by hand using pencil (or pen) and paper.
  There is no need to type of this assignment.
\item Make a pdf scan of your work using genius scan or some
  similar scanning app.  Please make the homework into a single
  pdf, not multiple pdfs. 
\item Submit the assignment on google classroom.  Please don't
  email it to me.  (Between my two classes I will be receiving
  over 40 assignments a week.  Keeping track of them all in email
  is challenging.)
\item If you want, you can do all the problems in pairs and hand
  in only one set of solutions.
\item If you work with someone else and they submit the solutions, it
  would be helpful for me if you submitted the assignment (without an
  attachment) on google classroom and mentioned who you worked with.
  Thanks. 

\end{itemize}


\noindent {\bf {\large IMPORTANT!!}}  Be sure you have the most
current version of the textbook.  Go the webpage and do a hard
refresh!  Sorry for yelling.\\ 


\noindent {\bf From the Tuesday Discussion Section}
Turn in the exercises from Tuesday's discussion section.  The link to
this assignment is here:
\url{http://hornacek.coa.edu/dave/Teaching/Energy.W21/pdf/wind_lab.pdf}.\\   

\noindent {\bf From Chapter 12}
\begin{enumerate}
\setlength{\itemsep}{-1mm}
\item 12.9
\item 12.10  
\item Optional: 12.15--12.18.  This problem leads you through a
  derivation of the Betz limit.  Requires some intro physics and a
  little bit of differential calculus at the end.  
\end{enumerate}










\end{document}
