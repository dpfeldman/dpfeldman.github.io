\documentclass[12pt]{article}
\oddsidemargin=0.0in
\evensidemargin=0.0in
\textwidth=6.5in
%\topmargin=-.75in
%\textheight=9in
%\usepackage{doublespace}

\begin{document}
\pagestyle{empty}

\begin{center}
{\large {\bf Calculus II}}\\
\medskip
{\large {\bf Homework Eight}}\\
\medskip
{ {\bf Due Friday 23 February, 2009}}\\
\end{center}

\hspace{2mm}\\

\noindent {\bf Chapter 8.8:}

\begin{enumerate}
\setlength{\itemsep}{-1mm}
\item 7
\item 10
\end{enumerate}

\noindent {\bf Normal Distributions}

\begin{enumerate}
\setlength{\itemsep}{-1mm}
\item The height of giraffes is distributed according to a normal
  distribution with a mean of $5.2$ and a standard deviation of
  $0.3$.  
\begin{enumerate}
\setlength{\itemsep}{-1mm}
\item What fraction of giraffes are less than 4 meters tall?
\item What fraction of giraffes are between 5 and 6 meters tall?
\item What fraction of giraffes are more than 5.5 meters tall?
\end{enumerate}
Answer these questions two ways:
\begin{itemize}
\setlength{\itemsep}{-1mm}
\item Using Maple to do the integrals
\item Converting to z and using a z-table.
\end{itemize}

\item Sarah Luke is interested in the heights of COA students compared
  to Hampshire students.  A careful study reveals that COA students
  have an average height of 63 inches.  Sarah then sends a team of RAs
  on a trip to Massachusetts to measure the heights of some Hampshire
  students.  The RA team manages to convince 25 Hampshire students to
  be measured.  The mean of these 25 Hampshire students is 67 inches.
  The standard deviation of this sample of Hampshire students is 3.
\begin{enumerate}
\setlength{\itemsep}{-1mm}
  \item What is the null hypothesis?
  \item What is the p-value?
  \item Should you reject the null?  
\end{enumerate}

\end{enumerate}



\end{document}
