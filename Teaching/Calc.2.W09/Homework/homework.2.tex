\documentclass[12pt]{article}
\oddsidemargin=0.0in
\evensidemargin=0.0in
\textwidth=6.5in
%\topmargin=-.75in
%\textheight=9in
%\usepackage{doublespace}
 
\begin{document}
\pagestyle{empty}
 
\begin{center}
{\large {\bf Calculus II}}\\
\medskip
{\large {\bf Homework Two}}\\
\medskip
{ {\bf Due Friday 16 February, 2009}}\\
\end{center}

\hspace{2mm}\\

\noindent {\bf Chapter 5.2:}

\begin{enumerate}
\setlength{\itemsep}{-1mm}
  \item 1
  \item 7
  \item 9
  \item 19
  \item 28
  \item 29

\end{enumerate}

\noindent {\bf Chapter 5.3:}

\begin{enumerate}
\setlength{\itemsep}{-1mm}
  \item 6
  \item 10
  \item 14
  \item 23

\end{enumerate}

\noindent {\bf Use Maple to do the following:}

\begin{enumerate}
\setlength{\itemsep}{-1mm}
  \item Determine the sum of the first $100$ integers.
  \item Determine the sum of the first $100$ squares. I.e., $1 + 4 + 9
  +16 + \cdots + (99)^2 + (100^2) \;$.
  \item Consider the following sum:
\begin{equation}
  \frac{4}{1} - \frac{4}{3} + \frac{4}{5} - \frac{4}{7} + \frac{4}{9}
  + \cdots \;.
\end{equation}
Determine the value of this sum if it include $10,000$ terms.  Determine
  the value ofthe sum if it include one million terms.  Does this
  number look familiar?
\end{enumerate}



%\noindent {\bf Chapter 5.4:}

%\begin{enumerate}
%\setlength{\itemsep}{-1mm}
%  \item 7
%  \item 14-18
%  \item 26
%  \item 34-35
%  \item 36-39
%
%\end{enumerate}


%\noindent {\bf Chapter 2 Review (p.~103):}

%\begin{enumerate}
%\setlength{\itemsep}{-1mm}
%  \item 1-6
%\end{enumerate}

%\noindent {\bf Chapter 3 Review (p.~159):}

%\begin{enumerate}
%\setlength{\itemsep}{-1mm}
%  \item 1-17, odd only
%\end{enumerate}


\end{document}
