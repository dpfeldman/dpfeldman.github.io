\documentclass[12pt]{article}
\oddsidemargin=0.0in
\evensidemargin=0.0in
\textwidth=6.5in
%\topmargin=-.75in
%\textheight=9in
%\usepackage{doublespace}
 
\begin{document}
\pagestyle{empty}
 
\begin{center}
{\large {\bf Calculus II}}\\
\medskip
{\large {\bf Homework One}}\\
\medskip
{ {\bf Due Friday 9 January, 2009}}\\
\end{center}

\hspace{2mm}\\

\noindent {\bf Chapter 5.1:}

\begin{enumerate}
\setlength{\itemsep}{-1mm}
  \item 1
  \item 7
  \item 8
  \item 17-19
  \item 25
  \item 26 

\end{enumerate}

\noindent {\bf Use Maple to do the following:}

\begin{enumerate}
\setlength{\itemsep}{-1mm}
  \item Determine the sum of the first $100$ integers.  
  \item Determine the sum of the first $100$ squares. I.e., $1 + 4 + 9
  +16 + \cdots + (99)^2 + (100^2) \;$.  
  \item Consider the following sum:
\begin{equation}
  \frac{4}{1} - \frac{4}{3} + \frac{4}{5} - \frac{4}{7} + \frac{4}{9}
  + \cdots \;.
\end{equation} 
Determine the value of this sum if it include $1000$ terms.  Determine
  the value ofthe sum if it include one million terms.  Does this
  number look familiar?
\end{enumerate}


\end{document}
