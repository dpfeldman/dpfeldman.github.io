
\documentstyle[12pt]{article}
\oddsidemargin=0in
\textwidth=6.75in
%\topmargin=-.75in
%\textheight=9in
%\usepackage{doublespace}

\renewcommand{\arraystretch}{1.3}

\begin{document}
\pagestyle{empty}

\begin{center}
{\large Lab 5:  Latent Heat, Specific Heat, and Rolling Stuff}\\
\end{center}
\bigskip

\begin{enumerate}

\item {\bf Rotational Kinetic Energy.} 

Consider a hoop and a disk, each with a diameter of 15 cm.  Suppose
they roll down a table that is tilted so that one end is 30 cm higher
than the other.  Find the velocity of the hoop and the disk at the
end of the incline.


\item {\bf Specific Heat}

\begin{enumerate}

\item Suppose you place a hot piece of metal in a styrofoam cup
containing some water. By how much will the temperature of the water
increase?  Try this out with one of the known pieces of metal.  Then
calculate the expected temperature rise as done in example 10.3 on
page 145.  


\item Repeat the experiment with another piece of metal.  Based on the
temperature rise you observe, what is the specific heat of the metal?  


\end{enumerate}


\item {\bf Latent Heat}

Let's calculate the latent heat of boiling for water.  Pour some
water into a beaker.  Measure the mass of the beaker with the water in
it.  Place the beaker on a hot plate and put a thermometer in the
water.  Measure the temperature of the water once each minute.  Record
the temperature readings.  Eventually the water will come to a boil.
Keep making the temperature readings for another 5 minutes or so.
Then, take the beaker off the flame.  Be careful not to burn
yourself.  Measure the mass of the beaker again.  It should be lighter
since a bunch of the water will have boiled away.  Based on how much
mass was boiled and a knowledge of how long the water was boiling, you
should be able to calculate the latent heat.


\end{enumerate}


\end{document}
