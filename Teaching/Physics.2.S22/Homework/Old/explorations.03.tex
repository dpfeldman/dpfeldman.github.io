\documentclass[12pt]{article}
\oddsidemargin=0.0in
\evensidemargin=0.0in
\textwidth=6.5in
\topmargin=-0.55in
\textheight=9in
%\usepackage{doublespace}
\usepackage{hyperref}

\hypersetup{
    colorlinks=true,
    linkcolor=magenta,
    filecolor=magenta,      
    urlcolor=blue,
    %pdftitle={Overleaf Example},
    %pdfpagemode=FullScreen,
    }

\begin{document}
\pagestyle{empty}
 
\begin{center}
{\LARGE {\bf Julia Set Exploration}}\\
\bigskip
{\Large {\bf Chaos and Fractals}}\\
\bigskip
{\Large {\bf College of the Atlantic}}\\
\bigskip
    {{\bf Please bring a Julia set to class, Monday Feb 21}}\\
\end{center}
\medskip

\noindent General Instructions

\begin{itemize}
\setlength{\itemsep}{0mm}
\item Do this with others if you like. It might be more fun that
  way.
\item Here are three pages you can use for making Julia sets:
  \begin{enumerate}
  \item
    \href{https://sciencedemos.org.uk/julia.php}{\url{https://sciencedemos.org.uk/julia.php}}. Less
    psychedelic.
    \item
      \href{https://www.kylepaulsen.com/stuff/juliaSet.html}{\url{https://www.kylepaulsen.com/stuff/juliaSet.html}}
      more psychedelic.  

    \item
      \href{http://shodor.org/interactivate/activities/JuliaSets/}{\url{http://shodor.org/interactivate/activities/JuliaSets/}} 
      Barebones but effective.
    
\end{enumerate}
\end{itemize}


\noindent The above programs let you visualize the Julia set (aka the
prisoner set) for functions of the form $f(z) = z^2 + c$.  In class we
used post-its to build the Julia set for the case where $c=-1$.
Different $c$ values yields a stunning diversity of Julia sets.\\


\begin{enumerate}
\item Experiment with one of the above programs until you find a
  Julia set that you like.
\item Write down the $c$ value for your Julia set.
\item Give your Julia set a name.
\end{enumerate}

\noindent We will start class on Monday by looking at your Julia sets.  We'll
then use this gallery of images to launch our explorations of the
Mandelbrot set. 

\end{document}
