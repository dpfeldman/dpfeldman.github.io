\documentclass[12pt]{article}
\oddsidemargin=0.0in
\evensidemargin=0.0in
\textwidth=6.5in
\topmargin=-0.35in
\textheight=9in
\usepackage{hyperref}
 
\begin{document}
\pagestyle{empty}
 
\begin{center}
{\LARGE {\bf Homework Four}}\\
\bigskip
{\Large {\bf Physics and Math of Sustainable Energy}}\\
\bigskip
{\Large {\bf College of the Atlantic}}\\
\bigskip
{ {\bf Due Friday, October 10, 2025}}\\ 
\end{center}
\medskip


\noindent There are two parts to this assignment.\\

\noindent {\bf Part 1: WeBWorK}.  Do Homework 04 which you
will find the WeBWorK page here: \url{https://webwork-hosting.runestone.academy/webwork2/coa-feldman-es1056i-fall2025}
I recommend doing the WeBWorK part of the homework first.  This will
enable you to benefit WeBWorK's instant feedback before you do part
two.\\ 


\noindent {\bf Part 2: Problems from the Textbook}.  Here are some
instructions for how to submit this part of the assignment.
\begin{itemize}
\item Do the problems by hand using pencil (or pen) and paper.
  There is no need to type this assignment.
\item If you like working on a tablet, go for it. 
\item Make a pdf scan of your work using genius scan or some
  similar scanning app.  Please make the homework into a single
  pdf, not multiple pdfs.
\item Submit the assignment on google classroom.  Please don't
  email it to me.  (Between my two classes I will be receiving
  around 60 assignments a week.  Keeping track of them all in email 
  is challenging.)\\
\end{itemize}

\noindent The problems you should do are from {\bf Chapter 18, 7, and
  8} of the book: \\

\noindent {\bf Chapter 18}
\begin{enumerate}
\setlength{\itemsep}{-1mm}
  \item 18.11
\end{enumerate}

\noindent {\bf Chapter 7}
\begin{enumerate}
\setlength{\itemsep}{-1mm}
%\item 18.11
\item 7.4
\end{enumerate}

\noindent {\bf Chapter 8}
\begin{enumerate}
\setlength{\itemsep}{-1mm}
\item 8.1
\item 8.2 (The data you need in in appendix A.)
\end{enumerate}

\noindent Information about how to access the book is on the
pinned post on google classroom. 



\end{document}
