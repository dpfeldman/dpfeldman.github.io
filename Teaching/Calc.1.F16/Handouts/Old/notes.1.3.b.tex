\documentclass[12pt]{article}
\oddsidemargin=-0.250in
\evensidemargin=-.250in
\textwidth=7.0in
\topmargin=-.35in
\textheight=9in

\renewcommand{\arraystretch}{1.3}

\begin{document}
\pagestyle{empty}
\begin{center}
{\Large {\bf Chapter 1.3: Stretching and Shifting}}\\
\medskip
{\large {\bf Calculus I}}\\
\medskip
{\bf College of the Atlantic.  Fall 2014}\\
\end{center}


\noindent Use the values for $g(x)$ given in the first table to
complete the second table.  

\begin{center}
\begin{tabular}{|| l | c ||}
\hline $x$ & $g(x)$ \\
\hline 
%-6 &  1 \\
-5 & 1\\
-4 & 1\\
-3 & 1\\
-2 & 2\\
-1 & 1\\
0 & 1\\
1 & 1\\
2 & -2\\
3 & 1\\
4 & 1\\
5 & 1\\
%6 & 1\\
\hline 
\end{tabular}
\end{center}

%Complete the following table:
\begin{center}
\begin{tabular}{|| l | c |c|c|c|c||}
\hline $x$ & $2g(x)$ & g(x+2) & g(x-2)& g(2x) & g(x/2) \\
\hline
-5 &  & & & & \\
-4 &  & & & & \\
-3 &  & & & & \\
-2 &  & & & & \\
-1 &  & & & & \\
0 &  & & & & \\
1 &  & & & & \\
2 &  & & & & \\
3 &  & & & & \\
4 &  & & & & \\
5 &  & & & & \\
%6 &  & & & & \\
\hline 
\end{tabular}
\end{center}
Sketch (on the same axes) the following functions using the table of
numbers you just made. 

\begin{enumerate}

\item $g(x)$ and $2g(x)$.

\item $g(x)$, $g(x+2)$, and $g(x-2)$

\item $g(x)$, $g(2x)$, and $g(x/2)$

\end{enumerate}
\newpage
Let $S(Q)$ give the fraction of TAB patrons consuming salads as a
function of the quality of the lunch entree.  Assume that the  
lunch quality $Q$ is measured on a scale of $1$ to $5$, with $5$
indicating yumminess and $1$ indicating in-edibility.  
\begin{enumerate}
  \setlength{\itemsep}{-1mm}
  \item Sketch a possible graph for $S(Q)$.
%  \item Sketch the inverse of $S(Q)$.
  \item What is the meaning of $S(2.2)$? 
  \item What is the meaning of $S(4.2) = 0.5$?
  \item What is the meaning of $S^{-1}(0.78) = 3.9$? 
  \end{enumerate}


\end{document}
\newpage

Using the graph paper provided and using the table of numbers you just
made, sketch the following functions.   

\begin{enumerate}

\item $g(x)$ and $2g(x)$.

\item $g(x)$, $g(x+2)$, and $g(x-2)$

\item $g(x)$, $g(2x)$, and $g(x/2)$

\end{enumerate}

\noindent If possible, sketch the different functions in different colors.  


 
% **********************************************************************
\begin{figure}[h]
\vspace{0.2in}
%\hspace{0.2in}
\epsfxsize=5.5in
\begin{center}
\leavevmode
\epsffile{grid.eps}
\end{center}
%\vspace{2mm}
%\caption{An unknown function.}
%\label{line}
\end{figure}
% **********************************************************************


% **********************************************************************
\begin{figure}[h]
\vspace{-0.2in}
%\hspace{0.2in}
\epsfxsize=5.25in
\begin{center}
\leavevmode
\epsffile{grid.eps}
\end{center}
%\vspace{2mm}
\end{figure}
% **********************************************************************


% **********************************************************************
\begin{figure}[hb]
\vspace{-0.2in}
%\hspace{0.2in}
\epsfxsize=5.25in
\begin{center}
\leavevmode
\epsffile{grid.eps}
\end{center}
%\vspace{2mm}
\end{figure}
% **********************************************************************


\end{document}


