\documentclass[12pt]{article}
\oddsidemargin=0.0in
\evensidemargin=0.0in
\textwidth=6.5in
%\topmargin=-.75in
%\textheight=9in
%\usepackage{doublespace}
 
\begin{document}
\pagestyle{empty}
 
\begin{center}
{\large {\bf Calculus II}}\\
\medskip
{\large {\bf Homework Five}}\\
\medskip
{ {\bf Due Friday 1 May, 2015}}\\
\end{center}

\hspace{2mm}\\
   
\noindent For the problems from chapters 7.3 and 7.4, you should feel
free to use WolframAlpha if you wish. For problems from 7.2 you can
use a program to check your work, but you should do any integrals by
hand and show your work.\\
\hspace{2mm}

\noindent {\bf Chapter 7.2:}

\begin{enumerate}
\setlength{\itemsep}{-1mm}
%  \item 5
%  \item 7
%  \item 9
  \item 11
  \item 12
  \item 31
  \item 33
  \item 46
  \item 56
\end{enumerate}

\noindent {\bf Chapter 7.3:}

\begin{enumerate}
\setlength{\itemsep}{-1mm}
  \item 17--20
  \item 41
  \item 42
\end{enumerate}


\noindent {\bf Chapter 7.4:}

\begin{enumerate}
\setlength{\itemsep}{-1mm}
  \item 33--35
  \item 63
  \item 64
\end{enumerate}
 Hint for 63 and 64: Wolframalpha will not do the definite integral for
    you.  But it will do an indefinite integral.  So use this to find
    the anti-derivative, then evaluate the limits to get the definite
    integral you need.


\end{document}
