\documentclass[12pt]{article}
\oddsidemargin=0.0in
\evensidemargin=0.0in
\textwidth=6.5in
\topmargin=-.35in
\textheight=8.4in
\usepackage{url}
\usepackage{hyperref}

\begin{document}
\pagestyle{empty}

\begin{center}
{\large {\bf Calculus II}}\\
\medskip
{\large {\bf Homework Eight}}\\
\medskip
{ {\bf Due May 22, 2015}}\\
\end{center}


\noindent {\bf Chapter 9.1:}

\begin{enumerate}
\setlength{\itemsep}{-1mm}
\item 2
\item 4
%\item 6
\item 8
\item 10
\item 14
\item 20--28, even only
\item 56 (Optional, but recommended for Fibonacci enthusiasts)
\end{enumerate}



\noindent {\bf Chapter 9.2:}

\begin{enumerate}
\setlength{\itemsep}{-1mm}
  \item 11
  \item 12
  \item 18
%  \item 19
  \item 20
%  \item 30
  \item 31
\end{enumerate}


%\noindent {\bf Chapter 8.7:}
%
%\begin{enumerate}
%\setlength{\itemsep}{-1mm}
%  \item 1--3
%  \item 11
%  \item 12
%  \item 17
%%  \item 20 
%\end{enumerate}
%
%\noindent {\bf Chapter 8.8:}
%\begin{enumerate}
%\setlength{\itemsep}{-1mm}
%  \item 6
%%  \item 7
%%  \item 10
%\end{enumerate}

\noindent {\bf Normal Distributions}

\begin{enumerate}
\setlength{\itemsep}{-1mm}
\item The height of giraffes is distributed according to a normal
  distribution with a mean of $5.2$ and a standard deviation of
  $0.6$.  
\begin{enumerate}
\setlength{\itemsep}{-1mm}
\item What fraction of giraffes are less than 4 meters tall?
\item What fraction of giraffes are between 5 and 6 meters tall?
\item What fraction of giraffes are more than 5.5 meters tall?
\end{enumerate}
Answer these questions two ways:
\begin{itemize}
\setlength{\itemsep}{-1mm}
\item Using WolframAlpha to evaluate the integrals.  You do not need
  to show printouts from wolfram alpha, but you should write down the
  integrals that you are asking wolfram alpha to solve for you.
\item Converting to z and using a z-table. See, e.g.,
  \url{www.stat.ufl.edu/~athienit/Tables/Ztable.pdf}.  Briefly explain
  how you used the z table to answer each of the questions. 
\end{itemize}

\item Sarah Luke is interested in the heights of COA students compared
  to Hampshire students.  A careful study reveals that COA students
  have an average height of 63 inches and a standard deviation of 4
  inches.  Sarah then sends a team of RAs on a trip to Massachusetts
  to measure the heights of some Hampshire students.  The RA team
  manages to convince 25 Hampshire students to be measured.  The mean
  of these 25 Hampshire students is 66 inches. 
  %The standard deviation of this sample of Hampshire students is 3.
\begin{enumerate}
\setlength{\itemsep}{-1mm}
  \item What is the null hypothesis?
  \item What is the p-value?
  \item Should you reject the null?  Do you think it is likely that
    the average heights of Hampshire and and COA students are the
    same? 
\end{enumerate}

\item Repeat the above question, but suppose that the RAs measured 100
  Hampshire students and found an average height of 66 inches.  

\end{enumerate}






\end{document}
