\documentstyle[12pt]{article}
\oddsidemargin=0in
\textwidth=6.75in
%\topmargin=-.75in
\textheight=7.5in
%\usepackage{doublespace}

\renewcommand{\arraystretch}{1.3}

\begin{document}
\pagestyle{empty}
\begin{center}
\bigskip
{\Large {\bf Equipment Acquisition Process }}\\
\bigskip
\bigskip
{\large {\bf A Proposal Submitted to }}\\
\bigskip
{\large {\bf The Academic Affairs Committee}}\\
\bigskip
{\large {\bf by}}\\
\bigskip
{\large {\bf The President's Task Force on}}\\
\smallskip
{\large {\bf Strategic Technology and Communications }}\\
\medskip
\bigskip
{\bf {\large 24 November 1999}}\\
\end{center}
\smallskip
\hspace{2mm}\\

\noindent {\bf Introduction} \\

The Strategic Plan of the College of the Atlantic articulates the
goals of developing ``an educational technology and communication
system to support collaborative, interdisciplinary teaching and
learning, and to bring the outside world to the College, and the
College to the World.''  In our collective work on technology issues
on campus the task force has consistently encountered a fundamental
problem which has hindered our ability to focus on the strategic
planning goals of the College with reference to Technology. 

     The way we currently acquire new equipment to support our
curriculum and the administrative needs of the College has
appeared again and again as a central concern of all College
constituencies in our work with them on technology issues.  Therefore,
the President's Task Force on Technology recommends to the Academic
Affairs Committee the following process for the open, clear, and fair
determination of equipment needs, and the subsequent rational budgetary
planning to provide those needs.  Without this process in place our
strategic goals on technology will constantly be held back by the fact
that faculty and staff equipment needs of the most basic kind are not
met by the current system. \\
\bigskip


\noindent {\bf Process Proposal}


\begin{itemize}
\item  Identify where funding for equipment comes from and how much
is available to meet our community needs.  Within the Academic
Programs and General Operations budget there are presently
several budget lines that meet this need:  The Equipment Line
(\$2,000/year); The Audio Visual Budget (dollar amount unknown); and
Capital Equipment budget (dollar amount unknown). 

\item Openly announce to faculty that equipment may be requested at the
beginning of each term by submitting a request to Academic Affairs in
writing by the second week of term.  All requests will be considered
simultaneously by a subcommittee of Academic Affairs and the faculty
will be informed if they have been given the monies.  One third of all
equipment money will be set aside for each term and will be disbursed
only by the informed consent of the Academic Affairs Equipment subcommittee.
          

\item Major equipment (over \$250.00) requests will be submitted on an
annual cycle in the fall for inclusion in the next year's budget.  These will
be submitted in writing to the same subcommittee of Academic Affairs.
The subcommittee will in turn submit an annual {\em prioritized}\/
equipment request for inclusion in the next year's budget to the
appropriate College budget-making group.  The sub-committee will also
refer appropriate requests to the Development Office for inclusion in
grant requests or for stand alone one time grant requests. \\

\end{itemize}

\noindent {\bf Possible Criteria for Prioritizing Equipment Requests} \\

We propose that the Academic Affairs Equipment Subcommittee use the
following guidelines when deciding how equipment funds will be
allocated. 

\begin{itemize}

\item Most generally, requests that will provide the greatest good for
the greatest number of students and/or faculty should be given
priority. 

\item Priority should be given to requests that have a direct
curricular impact and relation to academic programs.  Equipment should
be used primarily in classes or independent studies with students, not
for faculty members' research or creative activities. 

\item Collaborative requests from multiple faculty should be given
priority. 

\item The subcommittee should work to ensure that resources are
distributed equitably.  The subcommittee should, as much as possible,
make sure that the equipment budget isn't monopolized by any one
program, resource area, faculty member, etc.

%\item How recently a faculty has requested equipment should be taken
%into account.  These funds should be

\item Priority should be given to basic requests that can't be funded
elsewhere, i.e., there is no other place in the budget from which
funds could come.  \\

\end{itemize}

\newpage
\noindent {\bf Outstanding Issues}

\begin{itemize}

\item We need a real determination of equipment needs.  This is
essentially the academic equivalent of assessing our deferred
maintenance.  This would be an appropriate Title III line item to fill
institutional deficiencies 

\item Need to clarify who will be responsible for deciding what
equipment is bought and when.  There needs to be one open and fair
process. 

\item Clarify the destiny of indirect cost recovery funds vis a vis equipment needs    

\end{itemize}

\end{document}